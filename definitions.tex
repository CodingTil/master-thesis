% Theorems %
\newtheorem{theorem}{Theorem}
\numberwithin{theorem}{chapter}
\newtheorem{definition}{Definition}
\numberwithin{definition}{chapter}
\newtheorem{corollary}{Corollary}
\numberwithin{corollary}{chapter}
\newtheorem{proposition}{Proposition}
\numberwithin{proposition}{chapter}
\theoremstyle{remark}
\newtheorem{note}{Note}
\numberwithin{note}{chapter}

% Algorithm2e Keywords %
\SetKwBlock{Loop}{loop}{}
\SetKw{Break}{break}
\SetKw{None}{None}

% Functions &
\newcommand{\polyhedron}[1]{\mathcal{#1}}
\newcommand{\mat}[1]{{\bm{#1}}}
\renewcommand{\vec}[1]{{\bm{#1}}}
\newcommand{\conv}[0]{\operatorname{conv}}
\newcommand{\raycone}[0]{\operatorname{cone}}
\newcommand{\transpose}[0]{^\intercal}
\renewcommand{\setminus}{\backslash}
\newcommand{\rank}[0]{\operatorname{rank}}
\newcommand{\st}[0]{\operatorname{s.t.}}
\newcommand{\LP}[0]{\textit{LP}}
\newcommand{\DP}[0]{\textit{DP}}
\newcommand{\IP}[0]{\textit{IP}}
\newcommand{\MIP}[0]{\textit{MIP}}
\newcommand{\MP}[0]{\textit{MP}}
\newcommand{\LR}[1]{\textit{LR}\ifstrempty{#1}{}{\left({#1}\right)}}
\newcommand{\LDP}[0]{\textit{LDP}}
\newcommand{\RMP}[0]{\textit{RMP}}
\newcommand{\SP}[1]{{\textit{SP}\ifstrempty{#1}{}{^{#1}}}}
\newcommand{\RCP}[0]{\textit{RCP-SP}}
\newcommand{\FP}[0]{\textit{FP-SP}}
\newcommand{\SCIP}[0]{\texttt{SCIP}}
\newcommand{\CIP}[0]{\texttt{CIP}}
\newcommand{\GCG}[0]{\texttt{GCG}}
\newcommand{\indexset}[1]{\mathcal{#1}}

% Math %

\DeclarePairedDelimiter\abs{\lvert}{\rvert}%
\DeclarePairedDelimiter\norm{\lVert}{\rVert}%
\DeclarePairedDelimiter\floor{\lfloor}{\rfloor}
\DeclarePairedDelimiter\ceil{\lceil}{\rceil}

% Swap the definition of \abs* and \norm*, so that \abs
% and \norm resizes the size of the brackets, and the
% starred version does not.
\makeatletter
\let\oldabs\abs
\def\abs{\@ifstar{\oldabs}{\oldabs*}}
%
\let\oldnorm\norm
\def\norm{\@ifstar{\oldnorm}{\oldnorm*}}
\makeatother

\let\oldbar\bar
\newcommand{\ubar}[1]{{\underaccent{\oldbar}{#1}}}
\renewcommand{\bar}[1]{{\oldbar{#1}}}
\newcommand{\stkout}[1]{\ifmmode\text{\sout{\ensuremath{#1}}}\else\sout{#1}\fi}
