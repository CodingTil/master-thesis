\section{Branch-and-Price}\label{sec:cg_bp_bp}
In Section \ref{sec:cg_bp_ip}, we discussed reformulating an integer program into a master and pricing problem with a focus on integrality conditions. In this section, we explore solving an integer master program using column generation. Branching is essential when an optimal solution of the \LP{} relaxation has fractional values for the integer variables, making the solution infeasible for the \IP{}. To address this, we branch on these fractional variables, creating subproblems that explicitly exclude these fractional solutions. By recursively solving these subproblems, we eventually find an optimal integer solution, a process known as \textbf{branch-and-bound}.

In the context of column generation for integer master programs, we follow a similar approach: first, we relax the integrality constraints of the master problem, allowing us to solve the relaxation using column generation to optimality. Then, we check if the integrality conditions are satisfied. If not, we must cut off the fractional solution by branching, and re-optimize using column generation \cite{thebook}. This technique of combining branching with column generation is referred to as \textbf{branch-and-price}.

We have seen two distinct approaches to reformulating an \IP{} into an integer master and pricing problem: convexification (Section \ref{sec:cg_bp_ip_convexification}) and discretization (Section \ref{sec:cg_bp_ip_discretization}). Since integrality of the original variables is required in both approaches, we can always branch on fractional solutions of the original variables. However, discretization introduces additional integrality constraints on the master variables, which imply the integrality of the original variables. Therefore, in discretization, we can also branch on the master variables. In the following, we will discuss both approaches in more detail.

\subsection{Branching on the Original Variables}\label{sec:cg_bp_bp_branching_original}
Assume we have a fractional solution $\vec{x}^*_\RMP{}$ to the relaxed restricted master problem \RMP{}, i.e., there at least one integer variable $x_j$ for which $x_j^* \not\in \mathbb{Z}$. We now must cut off this fractional solution, for example by creating two subbranches (\textbf{dichotomous branching}), one where $x_j \leq \floor{x_j^*}$ and one where $x_j \geq \ceil{x_j^*}$. In each of these subtrees, a solution to \RMP{} should be guaranteed to only use columns that satisfy the branching decision, and during the solving process, the pricing problems should only be able to generate such columns.

\begin{note}
Branching on the original variables allows the subproblem to generate the interior points required for the correctness of the discretization approach, as discussed in Section \ref{sec:cg_bp_ip_discretization}.
\end{note}

In the branch-and-price context, there are two ways to enforce this branching decision \cite{thebook}:

\subsubsection{Branching in the Master Problem}
Recall that the \MP{} includes the following constraint:

\begin{equation}
\sum_{p \in P} \vec{x}_p \lambda_p + \sum_{r \in R} \vec{x}_r \lambda_r = \vec{x} \in \mathbb{Z}_+^n
\end{equation}

This constraint is now violated in the case of variable $x_j$. We can enforce the branching decision $x_j \leq \floor{x_j^*}$ by adding the following constraint to \MP{} (analogous for the up-branch):

\begin{equation}
\sum_{p \in P} x_{pj} \lambda_p + \sum_{r \in R} x_{rj} \lambda_r \leq \floor{x_j^*} \quad \left[{\color{blue} \alpha_j }\right]
\end{equation}

To continue generating only improving columns after branching, we must consider the dual variable $\alpha_j$ in the pricing problem:

\begin{equation}
\begin{aligned}
z^*_\SP{} = &\min & \left( \vec{c}\transpose - \vec{\pi}_\vec{b}\transpose \mat{A} \right) \vec{x} {\color{blue} - \alpha_j x_j} - \pi_0 & \\
&\st & \mat{D} \vec{x} & \geq \vec{d} \\
&& \vec{x} & \in \mathbb{Z}_+^n
\end{aligned}
\end{equation}

\subsubsection{Branching in the Pricing Problem}
Alternatively, we may add the branching decision directly to the pricing problem:

\begin{equation}
\begin{aligned}
z^*_\SP{} = &\min & \left( \vec{c}\transpose - \vec{\pi}_\vec{b}\transpose \mat{A} \right) \vec{x} - \pi_0 & \\
&\st & \mat{D} \vec{x} & \geq \vec{d} \\
&& {\color{blue} x_j} & {\color{blue} \leq \floor{x_j^*}}\\
&& \vec{x} & \in \mathbb{Z}_+^n
\end{aligned}
\end{equation}

However, \RMP{} might already contain generated columns that violate the branching decision. To ensure correctness of this implementation of the branching decision, we must forbid all existing columns with $x_j > \lfloor x_j^* \rfloor$ from being part of the solution in the master. This can be achieved by removing such columns altogether or by adding the following constraint to \MP{}:

\begin{equation}
\sum_{p \in P: x_{pj} > \floor{x_j^*}} \lambda_p + \sum_{r \in R: x_{rj} > \floor{x_j^*}} \lambda_r = 0
\end{equation}

\subsection{Branching on the Master Variables}\label{sec:cg_bp_bp_branching_master}
Let $Q \coloneqq \ddot{P} \cup R$. Assume our master solution $\vec{\lambda}^*_\RMP{}$ is fractional, i.e., $\lambda_q^* \not\in \mathbb{Z}$ for at least one $q \in Q$. Unfortunately, dichotomous branching on a such a single master variable $\lambda_q$ is very weak: Assume our problem consists of a single non-aggregated subproblem, and $\lambda_q^* = 0.5$. We then would create the down-branch $\lambda_q = 0$ and the up-branch $\lambda_q = 1$. The former constraint would cut off almost no solutions, while the latter would forbid most solutions. This would lead to an extremely unbalanced branching tree, which is only little better than enumerating all possible solutions \cite{thebook}. Cutting off multiple fractional solutions in each child node would be more desirable, i.e., we must branch on constraints, in general of the following form:

\begin{equation*}
\sum_{p \in \ddot{P}} \operatorname{f}(p) \lambda_p + \sum_{r \in R} \operatorname{f}(r) \lambda_r \leq f \qquad \left[\gamma\right]
\end{equation*}

Here, the function $\operatorname{f}$ determines the coefficient of column $p$ or $r$ in the constraint, and $f$ is a constant. The subproblem would have to respect the dual value $\gamma$ of the constraint in the pricing problem:

\begin{equation*}
\begin{aligned}
z^*_\SP{} = &\min & \left( \vec{c}\transpose - \vec{\pi}_\vec{b}\transpose \mat{A} \right) \vec{x} {\color{blue} - \gamma \operatorname{f}(\vec{x})} - \pi_0 & \\
&\st & \mat{D} \vec{x} & \geq \vec{d} \\
&& \vec{x} & \in \mathbb{Z}_+^n
\end{aligned}
\end{equation*}

The question one might ask themselves now is how we can find such a function $\operatorname{f}$ and constant $f$ that are suitable for branching. Henceforth, consider the case where $Q = \ddot{P}$, i.e., our problem is bounded ($R = \emptyset$). In this case, Vanderbeck proposes we can find a subset $\emptyset \subset Q' \subset Q$ of variables in \RMP{}, for which the following holds \cite{vanderbeck1996exact}:

\begin{equation}
\sum_{q \in Q'} \lambda_q^* \eqqcolon K \not\in \mathbb{Z}
\end{equation}

It is obvious that such a subset $Q'$ always exists, for example choose $Q' = \{\lambda_q\}$ for dichotomous branching. In the master problem, we could then branch on this integrality condition, e.g., in the down branch using:

\begin{equation}
\sum_{q \in Q'} \lambda_q \leq \floor{K} \quad \left[\gamma\right]
\end{equation}

The corresponding subproblem must be adapted to ensure the validity of the branching decision in the master. In particular, if and only if the pricing problem generates a new column $q'$ for which $q' \in Q'$, the corresponding master variable $\lambda_{q'}$ must be set to 1. This can be achieved by adding the following constraint to the pricing problem:

\begin{equation}\label{eq:cg_bp_branchmaster_pricingmod}
\begin{aligned}
z^*_\SP{} = &\min & \left( \vec{c}\transpose - \vec{\pi}_\vec{b}\transpose \mat{A} \right) \vec{x} {\color{blue} - \gamma y} - \pi_0 & \\
&\st & \mat{D} \vec{x} & \geq \vec{d} \\
&& {\color{blue} y = 1} & {\color{blue} \Leftrightarrow \vec{x} \in Q'} \\
&& \vec{x} & \in \mathbb{Z}_+^n \\
&& {\color{blue} y} & {\color{blue} \in \{0, 1\}}
\end{aligned}
\end{equation}

This idea Vanderbeck proposed plays nicely into our more general definition, where we set $\operatorname{f}(q) = \mathbbm{1}_{q \in Q'}$ and, in this down branch, $f = \floor{K}$.

What remains is to find a routine to determine such a subset $Q'$ in the master solution, for which the set inclusion rule to be added to \SP{} is also \textit{expressible using a finite set of linear constraints}.

\begin{note}
Adding the variable $y$ to the subproblem allows the column generation algorithm to generate the interior points required for the correctness of the discretization approach, as discussed in Section \ref{sec:cg_bp_ip_discretization} \cite{vanderbeck1996exact}.
\end{note}

\begin{note}
Aggregation of subproblems (Section \ref{sec:cg_bp_idsp}) is not an issue when branching in the master. In such cases, we would simply branch on the aggregated variables within each block of identical subproblems. For readability, we focus only on single non-aggregated blocks.
\end{note}

\subsubsection{Vanderbeck's Generic Branching Scheme}\label{sec:cg_bp_bp_branching_generic}
Vanderbeck proposed an elaborate scheme (\texttt{GENERIC}) \cite{vanderbeck2010reformulation,vanderbeck1996exact} to find such a subset $Q'$ in the master solution, enabling branching on master variables for any type of bounded \IP{}, i.e., which has no extreme rays ($Q = \ddot{P}$). This branching rule is based on component bounds on original variables:

\begin{equation}
B \coloneqq \left( x_i, \eta, v \right) \in \{x_i \mid 1 \leq i \leq n\} \times \{\leq, \geq\} \times \mathbb{Z}
\end{equation}
\begin{equation}
\bar{B} \coloneqq \left( x_i, \bar{\eta}, v \right), \bar{\eta} \coloneqq \begin{cases} \leq & \text{if } \eta = \geq \\ \geq & \text{if } \eta = \leq \end{cases}
\end{equation}
where $\eta$ is the type of bound, and $v$ is the value of the bound. Furthermore, $\bar{B}$ describes the inverse component bound of $B$. We can now define a component bound sequence as:

\begin{equation}
S \coloneqq \left\{ \left( x_{i,1}, \eta_1, v_1 \right), \dots, \left( x_{j,k}, \eta_k, v_k \right) \right\} \in 2^{\{x_i \mid 1 \leq i \leq n\} \times \{\leq, \geq\} \times \mathbb{Z}}
\end{equation}

Let us further introduce the following shorthand notation:

\begin{equation}
\eta(a, v) \Leftrightarrow
\begin{cases}
a \leq v & \text{if } \eta = \leq \\
a \geq v & \text{if } \eta = \geq
\end{cases}
\end{equation}

For a given component bound sequence $S$ and a set of columns $Q$, we can define the restriction of $Q$ to $S$ as:

\begin{equation}
Q(S) \coloneqq \left\{ q \in Q \mid \forall \left( x_i, \eta, v \right) \in S. \eta(x_{qi}, v) \right\}
\end{equation}
Note that $Q(\emptyset) = Q$.

We now reduce the problem of finding a subset $Q'$ to finding a component bound sequence $S$ for which the following holds:

\begin{itemize}
\item $\sum_{q \in Q(S)} \lambda_q^* \eqqcolon K \not\in \mathbb{Z}$
\item $y = 1 \Leftrightarrow \vec{x} \in Q(S)$ is expressible using a finite set of linear constraints
\end{itemize}

\begin{proposition}
If $\vec{\lambda}^*_\RMP{}$ is a fractional solution to the master problem, then there exists a component bound sequence $S$ for which the first condition holds.
\end{proposition}

\begin{proof}\label{pr:cg_bp_bp}
Let $Q_{fraq} \coloneqq \{ q \in Q \mid \lambda_q^* \not\in \mathbb{Z} \} \neq \emptyset$ be the set of columns with currently fractional master variables. Then take $q* \coloneqq \arg\min_{q \in Q_{fraq}} \vec{x}_q$ as any minimal undominated column in $Q_{fraq}$.
From $q*$, we can now construct a component bound sequence $S$, which is only satisfied by $q*$ out of all $q \in Q_{fraq}$, as follows:
\begin{equation}
S \coloneqq \left\{ \left( x_i, \leq, \floor{x_{q*}} \right) \mid x_i \in \{x_j \mid q_j \in Q_{fraq}\} \right\}
\end{equation}
By construction, $Q(S) = \{q*\}$, and thus $\sum_{q \in Q(S)} \lambda_q^* = \lambda_{q*}^* \not\in \mathbb{Z}$.
\end{proof}

Vanderbeck's scheme divides the solution space along the component bounds into multiple sub-polyhedra. In this way, each child branch can only generate points within its own sub-polyhedron, and the master solution will be integral within one of these sub-polyhedra. In fact, this scheme closely resembles the main idea of dichotomous branching in branch-and-bound, where the solution space is divided into two halves. For a given component bound sequence $S = \{B_1, \dots, B_m\}$, where each variable $x_i$ has at least one upper and one lower component bound, there are up to $2^n - 1$ possible sub-polyhedra. To avoid an exponential increases in nodes, we group some sub-polyhedra together, creating a total of $n + 1$ nodes. Each of the $1 \leq j \leq m+1$ nodes is now modified as follows: first define the component bound sequence $S_j$ for the $j$-th node as:

\begin{equation}
S_j \coloneqq
\begin{cases}
\{B_1, \dots, B_{j-1}, \bar{B_j}\} & \text{if } j \leq m \\
\{B_1, \dots, B_m\} & \text{if } j = m+1
\end{cases}
\end{equation}

Determine the fractional value $K_j$ for the $j$-th node as:

\begin{equation}
K_j \coloneqq \sum_{q \in Q(S_j)} \lambda_q^*
\end{equation}

Then, to \RMP{} of node $j$, add the following constraint:

\begin{equation}
\sum_{q \in Q(S_j)} \lambda_q \geq \ceil{K_j} \quad \left[\gamma_j\right]
\end{equation}

Finally, modify the pricing problem as follows:

\begin{equation}
\begin{aligned}
z^*_\SP{} = &\min & \left( \vec{c}\transpose - \vec{\pi}_\vec{b}\transpose \mat{A} \right) \vec{x} {\color{blue} - \gamma_j} - \pi_0 & \\
&\st & \mat{D} \vec{x} & \geq \vec{d} \\
&& {\color{blue} x_i} & {\color{blue} \leq v} & {\color{blue} \forall \left( x_i, \leq, v \right) \in S_j} \\
&& {\color{blue} x_i} & {\color{blue} \geq v} & {\color{blue} \forall \left( x_i, \geq, v \right) \in S_j} \\
&& \vec{x} & \in \mathbb{Z}_+^n
\end{aligned}
\end{equation}

\begin{note}
The modifications made to the pricing problems during Vanderbeck's generic branching still fit the description stated in Equation \eqref{eq:cg_bp_branchmaster_pricingmod} and can be written formally as:

\begin{equation}
\begin{aligned}
z^*_\SP{} = &\min & \left( \vec{c}\transpose - \vec{\pi}_\vec{b}\transpose \mat{A} \right) \vec{x} {\color{blue} - \gamma_j y} - \pi_0 & \\
&\st & \mat{D} \vec{x} & \geq \vec{d} \\
&& {\color{blue} x_i} & {\color{blue} \leq v} & {\color{blue} \forall \left( x_i, \leq, v \right) \in S_j} \\
&& {\color{blue} x_i} & {\color{blue} \geq v} & {\color{blue} \forall \left( x_i, \geq, v \right) \in S_j} \\
&& {\color{blue} y} & {\color{blue} = 1} \\
&& \vec{x} & \in \mathbb{Z}_+^n \\
&& {\color{blue} y} & {\color{blue} \in \{0, 1\}}
\end{aligned}
\end{equation}
\end{note}

The procedure for finding a component bound sequence $S$ as described in Proof \ref{pr:cg_bp_bp} leads to dichotomous branching. As discussed earlier, branching on a single master variable leads to an unbalanced tree. To overcome this, Vanderbeck proposes a sophisticated routine that divides the solution space into multiple branches more evenly \cite{vanderbeck2011branching}.

This presentation of Vanderbeck's generic branching scheme just covers the main ideas and concepts. For a more in-depth derivation of this rule, detailed descriptions of the routines, and further optimizations such as node pruning, we refer to \cite{vanderbeck1996exact, vanderbeck2010reformulation, vanderbeck2011branching, schmickerath2012experiments}.
