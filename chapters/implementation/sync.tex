\section{Mastervariable Synchronization}\label{sec:implementation_sync}
As discussed in Section \ref{sec:gm_sync}, there is a need to synchronize the information of newly generated columns across the entire search tree. This requirement arose in the context of branching using generic mastercuts, but it is generally necessary for any branching rule that does not formulate its decisions in the original problem. Therefore, we have decoupled the synchronization of master variables from the generic mastercut interface and implemented it as a separate internal module within \GCG{}.

\begin{lstlisting}[language=C, caption=Variable History Construct]
struct GCG_VARHISTORYBUFFER {
	SCIP_VAR*             vars[50];
	int                   nvars;
	GCG_VARHISTORYBUFFER* next;
	int                   nuses;
};

struct GCG_VARHISTORY {
	GCG_VARHISTORYBUFFER* buffer;
	int                   pos;
};
\end{lstlisting}

To enable such mastervariable synchronization, we have implemented the history tracking approach using unrolled linked lists as described in Section \ref{subsec:gm_sync_history_unrolled}. Each element, or buffer, in the unrolled linked list has a default capacity of 50 variables. The variable \texttt{nuses}, which acts as a reference count, keeps track of how many strong pointers of type \texttt{GCG\_VARHISTORY} are currently pointing to the buffer. The \texttt{GCG\_VARHISTORY} structure is a simple wrapper around the buffer, keeping track of the current position in the buffer. Consequently, each strong pointer still points to a specific variable. The \GCG{} pricer is responsible for managing the global variable history list and appending new variables whenever a new column is generated. In Section \ref{sec:gm_sync}, we have denoted this central reference as the \texttt{Latest} pointer.

We can now synchronize master variables by attaching a strong reference to each node in the search tree. Specifically, upon node creation, we create a reference identical to the \texttt{Latest} pointer stored in the \GCG{} pricer. Then, following the procedure described in Section \ref{subsec:gm_sync_history}, we forward these strong pointers upon node (de-)activation. To inform a branching rule in a specific node about the creation of a new variable, we have extended the \GCG{} branching rule interface with the following callback function. The branching rule can then, for example, determine the constraint coefficient for this new master variable.

\begin{lstlisting}[language=C, caption=Branching Rule Interface Extension]
#define GCG_DECL_BRANCHNEWCOL(x) SCIP_RETCODE x (SCIP* scip, GCG_BRANCHDATA* branchdata, SCIP_VAR* mastervar)
\end{lstlisting}

Finally, we note that each \texttt{SCIP\_VAR} already contains a flag indicating whether it has been deleted from the problem. We use this flag instead of marking such variables as deleted in the history buffer ourselves.
