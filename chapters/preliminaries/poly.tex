\section{Polyhedron Representation}\label{sec:preliminaries_poly}

% Conic
\begin{definition}
Given $k$ points $\vec{x}_1, \dots, \vec{x}_k \in \mathbb{R}^n$, any $\vec{x} = \sum_{i=1}^{k} \alpha_i \vec{x}_i$ is a \textbf{conic combination} of the $\vec{x}_i$, iff $\forall i \in \{1, \dots, k\}. \alpha_i \geq 0$.
\end{definition}

% Convexity
\begin{definition}\label{def:convex}
Given $k$ points $\vec{x}_1, \dots, \vec{x}_k \in \mathbb{R}^n$, any $\vec{x} = \sum_{i=1}^{k} \alpha_i \vec{x}_i$ is a \textbf{convex combination} of the $\vec{x}_i$, iff $\sum_{i=1}^{k} \alpha_i = 1 \land \forall i \in \{1, \dots, k\}. \alpha_i \geq 0$.

The set of all convex combinations of $\vec{x}_1, \dots, \vec{x}_k$ is therefore defined as:
\begin{equation*}
\conv(\vec{x}_1, \dots, \vec{x}_k) \coloneqq \{\sum_{i=1}^{k} \alpha_i \vec{x}_i \mid \sum_{i=1}^{k} \alpha_i = 1 \land \forall i \in \{1, \dots, k\}. \alpha_i \geq 0\}
\end{equation*}
\end{definition}

\begin{corollary}\label{cor:intersection_convex}
The intersection of two convex sets is convex.
\end{corollary}

% Extreme Points
\begin{definition}
Let $\polyhedron{P}$ be a convex set. A point $\vec{p} \in \polyhedron{P}$ is an \textbf{extreme point} of $\polyhedron{P}$ if there is no non-trivial convex combination of any two points in $\polyhedron{P}$ expressing $\vec{p}$, i.e.
\begin{equation*}
\forall \vec{x}_1, \vec{x}_2 \in \polyhedron{P}. \forall \alpha \in \mathbb{R}_+ \setminus \{0\}. \vec{x}_1 \neq \vec{x}_2 \implies \vec{p} \neq \alpha \vec{x}_1 + (1 - \alpha) \vec{x}_2
\end{equation*}
\end{definition}

% Rays
\begin{definition}\label{def:rays}
Let $\polyhedron{P}$ be a convex set. A vector $\vec{r} \in \mathbb{R}_0^n \setminus \{0\}$ is a \textbf{ray} of $\polyhedron{P}$ iff $\forall \vec{x} \in \polyhedron{P}. \forall \beta \in \mathbb{R}_+. \vec{x} + \beta \vec{r} \in \polyhedron{P}$.

The span of rays $\vec{r}_1, \dots, \vec{r}_k \in \mathbb{R}_+^n$ we denote as:
\begin{equation*}
\rayspan(\vec{r}_1, \dots, \vec{r}_k) \coloneqq \bigcup_{i=1}^{k} \{\omega \vec{r}_i \mid \omega \in \mathbb{R}_+\}
\end{equation*}
\end{definition}

\begin{definition}
A ray $\vec{r}$ of $\polyhedron{P}$ is an \textbf{extreme ray} of $\polyhedron{P}$ if there is no non-trivial conic combination of any two rays in $\polyhedron{P}$ expressing $\vec{r}$, i.e.
\begin{equation*}
\forall \vec{r}_1, \vec{r}_2 \in \polyhedron{P}. \forall \alpha_1, \alpha_2, \beta \in \mathbb{R}_+ \setminus \{0\}. \vec{r}_1 \neq \beta \vec{r}_2 \implies \vec{r} \neq \alpha_1 \vec{r}_1 + \alpha_2 \vec{r}_2
\end{equation*}
\end{definition}

% Hyperplane
\begin{definition}
A \textbf{hyperplane} $\polyhedron{H} \subset \mathbb{R}^n$ of a $n$-dimensional space is a subspace of dimension $n-1$, and can therefore be described using a vector $\vec{f} \in \mathbb{R}^n$ and a scalar $f \in \mathbb{R}$ as $\polyhedron{H} = \{\vec{x} \mid \vec{f}\transpose \vec{x} = f\}$.
\end{definition}

\begin{corollary}
Any hyperplane is a convex set.
\end{corollary}

\begin{proof}
Let $\polyhedron{H} = \{\vec{x} \mid \vec{f}\transpose \vec{x} = f\}$ be a hyperplane. Let $k \in \mathbb{N}$, $\vec{x}_1, \dots, \vec{x}_k \in \polyhedron{H}$. For any $\alpha_1, \dots, \alpha_k \in \mathbb{R}_+$ with $\sum_{i=1}^{k}$:
\begin{align*}
\vec{f}\transpose \left( \sum_{i=1}^{k} \alpha_i \vec{x}_i \right)
&= \sum_{i=1}^{k} \alpha_i \vec{f}\transpose \vec{x}_i \\
&= \sum_{i=1}^{k} \alpha_i \cdot f \\
&= f \cdot \sum_{i=1}^{k} \alpha_i \\
&= f
\end{align*}
Therefore, the convex combination $\sum_{i=1}^{k} \alpha_i \vec{x}_i$ is in the hyperplane $\polyhedron{H}$.
\end{proof}

% Halfspace
\begin{definition}
A \textbf{halfspace} is the set above or below a hyperplane. A halfspace is open if the points on the hyperplane are excluded, otherwise closed.
\end{definition}

\begin{corollary}\label{cor:halfspace_convex}
Any halfspace is a convex set.
\end{corollary}

\begin{proof}
Let $\polyhedron{H}^+ = \{\vec{x} \mid \vec{f}\transpose \vec{x} > f\}$ be an open halfspace (analogous for $\polyhedron{H}^- = \{\vec{x} \mid \vec{f}\transpose \vec{x} < f\}$, and for the closed halfspaces). Let $k \in \mathbb{N}$, $\vec{x_1}, \dots, \vec{x_k} \in \polyhedron{H}$. For any $\alpha_1, \dots, \alpha_k \in \mathbb{R}_+$ with $\sum_{i=1}^{k}$:
\begin{align*}
\vec{f}\transpose \left( \sum_{i=1}^{k} \alpha_i \vec{x}_i \right)
&= \sum_{i=1}^{k} \alpha_i \vec{f}\transpose \vec{x}_i \\
&> \sum_{i=1}^{k} \alpha_i \cdot f \\
&= f \cdot \sum_{i=1}^{k} \alpha_i \\
&= f
\end{align*}
Therefore, the convex combination $\sum_{i=1}^{k} \alpha_i \vec{x}_i$ is in the halfspace $\polyhedron{H}$.
\end{proof}

% Polyhedron
\begin{definition}
A \textbf{polyhedron} $\polyhedron{P} \subseteq \mathbb{R}^n$ is defined by the intersection of a set of closed halfspaces, i.e. $\polyhedron{P} \coloneqq \{\vec{x} \in \mathbb{R}^n \mid \mat{A} \vec{x} \geq \vec{b}\}$, with $\mat{A} \in \mathbb{R}^{m \times n}, \vec{b} \in \mathbb{R}^m$.

By Corollaries \ref{cor:intersection_convex} and \ref{cor:halfspace_convex}, a polyhedron is also a convex set of points.
\end{definition}

% Ray

% Monkowski-Weyl
\begin{definition}
The \textbf{Minkowski sum} of two sets $P, Q$ is defined by:
\begin{equation*}
P \oplus Q \coloneqq \{\vec{p} + \vec{q} \mid \vec{p} \in P \land \vec{q} \in Q\}
\end{equation*}
\end{definition}

\begin{theorem}[Minkowski-Weyl]\label{th:minkowski-weyl}
For $\polyhedron{P} \subseteq \mathbb{R}^n$ the following statements are equivalent:
\begin{enumerate}
\item $\polyhedron{P}$ is a polyhedron, i.e., there exists some finite matrix $\mat{A} \in \mathbb{R}^{m \times n}$ and some vector $\vec{b} \in \mathbb{R}^m$ such that $P = \{\vec{x} \in \mathbb{R}^n \mid \mat{A} \vec{x} \leq \vec{b}\}$
\item There exist fine vectors $\vec{v}_1, \dots, \vec{v}_s \in \mathbb{R}^n$ and finite vectors $\vec{r}_1, \dots, \vec{r}_t \in \mathbb{R}_+^n$, such that $P = \conv(\vec{v}_1, \dots, \vec{v}_s) \oplus \rayspan(\vec{r}_1, \dots, \vec{r}_t)$
\end{enumerate}
\end{theorem}

In simple terms, the Minkowski-Weyl theorem states that any polyhedron can always be defined in two ways: either by its faces, i.e. closed halfspaces, or by its vertices and rays. Because of their unique properties, for such representation of a polyhedron it is sufficient to select its the extreme points and extreme rays. Figure xyz. illustrates this TODO-til

The following theorem builds upon the Minkowski-Weyl theorem to describe a polyhedron, which is represented by its extreme points $\{\vec{x}_p\}_{p \in P}$ and extreme rays $\{\vec{x}_r\}_{r \in R}$, using hyperplanes. Here, the sets $P, R$ are used to index the extreme points and extreme rays, respectively.

\begin{theorem}[Nemhauser-Wolsey]\label{th:nemhauser-wolsey}
Consider the polyhedron $\polyhedron{P} = \{\vec{x} \in \mathbb{R}^n \mid \mat{Q} \vec{x} \geq \vec{b}\}$ with full row rank matrix $\mat{Q} \in \mathbb{R}^{m \times n}$, i.e. $\rank(\mat{Q}) = m \leq n \land \polyhedron{P} \neq \emptyset$.
An equivalent description of $\polyhedron{P}$ using its extreme points $\{\vec{x}_p\}_{p \in P}$ and extreme rays $\{\vec{x}_r\}_{r \in R}$ is:
\begin{equation}
\polyhedron{P} = \left\{ \vec{x} \in \mathbb{R}^n \middle\vert
\begin{aligned}
\sum_{p \in P} \vec{x}_p \lambda_p &+ &\sum_{r \in R} \vec{x}_r \lambda_r &= \vec{x} &\\
\sum_{p \in P} \lambda_p & & &= 1 &\\
\lambda_p & & &\geq 0 &\forall p \in P\\
& &\lambda_r &\geq 0 &\forall r \in R
\end{aligned}
\right\}
\end{equation}
\end{theorem}

In the Nemhauser-Wolsey theorem, the conditions of the Minkowski-Weyl theorem are clearly encoded: the second and third lines ensure that the convex set of the extreme points are considered in the first line (Definition \ref{def:convex}), the last playing a part in the span of extreme rays (Definition \ref{def:rays}), and the first line being the Minkowski sum of the convex hull of extreme rays and the span of extreme rays.
