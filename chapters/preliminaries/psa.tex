\section{Primal Simplex Algorithm}\label{sec:preliminaries_psa}
Consider the following linear program in standard form:
\begin{equation}\label{eq:lp_standard}
\begin{aligned}
&\min & \vec{c}\transpose \vec{x} & & \\
&\st & \mat{A} \vec{x} & = \vec{b} & \left[\vec{\pi}\right] \\
&& \vec{x} & \geq \vec{0} &
\end{aligned}
\end{equation}

The primal simplex algorithm \cite{dantzig1997simplex} finds an optimal solution by moving from one extreme point of the polyhedron to the next, therefore always remaining feasible. A central aspect of this algorithm is the sufficient optimality condition. For a basic solution $\vec{X} = \left[\vec{x}_\indexset{B}, \vec{x}_\indexset{N}\right]$ at a given extreme point to be optimal, the reduced costs $\bar{c}_j \coloneqq c_j - \vec{\pi}\transpose \vec{a}_j$ for $j \in \indexset{N}$ must be non-negative.

This sufficient optimality condition leads to the \textbf{pricing problem}, which either verifies the optimality of the current basic solution or identifies the non-basic variable $x_l$, $l \in \indexset{N}$, with the least reduced cost ($\bar{c}_l < 0$) to be introduced into the basis next, following Dantzig's rule \cite{dantzig1997simplex,ploskas2014pivoting}. Formally, this is expressed:
\begin{equation*}
l \in \underset{j \in \indexset{N}}{\arg\min} \, c_j - \vec{\pi}\transpose \vec{a}_j
\end{equation*}
or as the linear program:
\begin{equation}\label{eq:psa_pp}
\bar{c}(\vec{\pi}) = \underset{j \in \indexset{N}}{\min} \, c_j - \vec{\pi}\transpose \vec{a}_j
\end{equation}

Solving the pricing problem is integral the primal simplex algorithm:

\begin{algorithm}
\caption{Primal simplex algorithm with Dantzig's rule}
\KwIn{\LP{} in standard form (\ref{eq:lp_standard}); Basic and non-basic index-sets $\indexset{B},\indexset{N}$}
\KwOut{Optimal Solution $(\vec{x}, z)$}
\Loop{
	$\vec{\pi}\transpose \gets \vec{c}_\indexset{B}\transpose \mat{A}_\indexset{B}^{-1}$;
	$\bar{\vec{b}} \gets \mat{A}_\indexset{B}^{-1} \vec{b}$\;
	$\bar{c}_j \gets c_j - \vec{\pi}\transpose \vec{a}_j$;$ \qquad \forall j \in \indexset{N}$\\
	$l \gets \underset{j \in \indexset{N}}{\arg\min} \, \bar{c}_j$;
	$\bar{c}(\vec{\pi}) \gets \bar{c}_l$\;
	\If{$\bar{c}(\vec{\pi}) \geq 0$}{
		\Return{$\left(\left[\bar{\vec{b}}, \vec{0}\right], \vec{c}_\indexset{B}\transpose \vec{x}_\indexset{B}\right)$} \textit{by optimality}
	}
	$\bar{\vec{a}}_l \gets \mat{A}_\indexset{B}^{-1} \vec{a}_l$\;
	\If{$\bar{\vec{a}}_l \leq \vec{0}$}{
		\Return{\None} \textit{by unboundedness}
	}
	$s \gets \underset{i \in \{1, \dots, m\}}{\arg\min} \, \frac{\bar{b}_i}{\bar{a}_{il}}$;
	$x_l \gets \frac{\bar{b}_s}{\bar{a}_{sl}}$;
	$\indexset{B} \gets \indexset{B} \cup \{l\} \subseteq \{s\}$;
	$\indexset{N} \gets \indexset{N} \cup \{s\} \subseteq \{l\}$\;
}
\end{algorithm}
