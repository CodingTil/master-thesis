\chapter{Conclusion}\label{ch:conclusion}
In this thesis, we have formalized and explored constraints that exist solely within the reformulated problem in column generation, where their validity in the master problem is ensured only by specific modifications to the pricing problem. To address these needs, we have extended the \GCG{} solver by introducing an interface that facilitates the creation and management of these specialized constraints, termed \texttt{generic mastercuts}. This new interface is designed to streamline the implementation of advanced branching rules and separators that rely on conditions specific to the reformulated problem.

Leveraging this interface, we have implemented the component bound branching rule, as presented by Desrosiers et al. \cite{thebook}. This rule offers a simpler alternative to Vanderbeck's generic branching scheme \cite{vanderbeck2011branching} for branching on component bound sequences. Our evaluation reveals that, while Vanderbeck's scheme significantly outperforms the new component bound branching rule, this advantage is largely due to the preservation of the pricing problem's structure. This preservation allows the continued use of specialized optimization algorithms throughout the entire search tree. When Vanderbeck's scheme is forced to rely on a generic \MIP{} solver for pricing, as is required by the component bound branching rule, the latter actually performs better on average. This finding underscores the critical importance of maintaining the structure of the pricing problem to enhance the performance of the branch-and-price algorithm, thereby emphasizing the need for careful selection of the decomposition strategy.

Although the component bound branching rule does not substantially improve the overall performance of \GCG{}, its implementation highlights the versatility and potential of the new generic mastercut interface. This interface makes it feasible to implement complex branching rules or separators that operate exclusively within the reformulated problem. As a further refinement, expanding the interface to accommodate constraints that do not require additional variables in the pricing problem, such as those in Vanderbeck's generic branching, would broaden its applicability and utility.

Looking ahead, future research could focus on developing new branching rules and master separators that can fully exploit the capabilities of this interface. This work will likely open up new avenues for improving the efficiency and effectiveness of branch-price-and-cut based algorithms in a wider range of applications. Moreover, given the importance of choosing the right component bound sequence for branching, as evidenced by our evaluation, further work could explore more effective heuristics for identifying these sequences. Investigating whether a theoretically optimal component bound sequence exists or developing more sophisticated heuristics could significantly enhance the performance of the component bound branching rule and, by extension, the efficiency of branch-and-price algorithms in solving large-scale integer programs.
