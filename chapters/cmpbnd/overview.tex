\section{Overview of the branching scheme}\label{sec:cmpbnd_overview}
Recall from Section \ref{sec:cg_bp_bp_branching_master}, given a fractional master solution $\vec{\lambda}^*_\RMP{}$, we can always find a subset $\emptyset \subset Q' \subset Q \coloneqq \ddot{P}$ for which the following holds:
\begin{equation}
\sum_{q \in Q'} \lambda_q^* \eqqcolon K \not\in \mathbb{Z}
\end{equation}
and we therefore can eventually enforce integrality of $\vec{\lambda}_\MP{}$, for example by adding one of the following branching constraints to each child node:
\begin{equation}
\begin{aligned}
\sum_{q \in Q'} \lambda_q \leq \lfloor K \rfloor \quad \left[\gamma\right] \\
\sum_{q \in Q'} \lambda_q \geq \lceil K \rceil \quad \left[\gamma\right]
\end{aligned}
\end{equation}

Adding such constraints to the master problem requires us to modify the pricing problem in the following way:
\begin{equation}
\begin{aligned}
z^*_\SP{} = &\min & \left( \vec{c}\transpose - \vec{\pi}_\vec{b}\transpose \mat{A} \right) \vec{x} {\color{blue} - \gamma y} - \pi_0 & \\
&\st & \mat{D} \vec{x} & \geq \vec{d} \\
&& {\color{blue} y = 1} & {\color{blue} \Leftrightarrow \vec{x} \in Q'} \\
&& \vec{x} & \in \mathbb{Z}_+^n \\
&& {\color{blue} y} & {\color{blue} \in \{0, 1\}}
\end{aligned}
\end{equation}
where $y$ becomes the column entry for the row added to the master, and $y = 1 \Leftrightarrow \vec{x} \in Q'$ is expressible using a finite set of linear constraints.

For finding such a $Q'$, which is expressible in the \SP{}, Vanderbeck uses bounds on the original variables (Section \ref{sec:cg_bp_bp_branching_generic}). Similar, we introduce a variation of component bounds on the original variables. Notably, instead of allowing fractional bounds, we now require integral bound values, and therefore enabling us to use $\geq$ instead of $>$.
\begin{equation}
B \coloneqq \left( x_i, \eta, v \right) \in \{x_i \mid 1 \leq i \leq n\} \times \{\leq, geq\} \times \mathbb{Z}
\end{equation}
\begin{equation}
\bar{B} \coloneqq \left( x_i, \bar{\eta}, v \right), \bar{\eta} \coloneqq \begin{cases} \leq & \text{if } \eta = \geq \\ \geq & \text{if } \eta = \leq \end{cases}
\end{equation}

Again, we define a component bound sequence:
\begin{equation}
S \coloneqq \left\{ \left( x_{i,1}, \eta_1, v_1 \right), \dots, \left( x_{j,k}, \eta_k, v_k \right) \right\} \in 2^{\{x_i \mid 1 \leq i \leq n\} \times \{\leq, geq\} \times \mathbb{Z}}
\end{equation}
as well as restrictions of $S$ to only upper bounds $\bar{S}$ and lower bounds $\ubar{S}$ respectively:
\begin{equation}
\begin{aligned}
\bar{S} &\coloneqq \left\{ \left( x_{i}, \leq, v \right) \mid \left( x_{i}, \leq, v \right) \in S \right\} \\
\ubar{S} &\coloneqq \left\{ \left( x_{i}, \geq, v \right) \mid \left( x_{i}, \geq, v \right) \in S \right\}
\end{aligned}
\end{equation}

We now redefine the following shorthand notation:
\begin{equation}
\eta(a, v) \Leftrightarrow
\begin{cases}
a \leq v & \text{if } \eta = \leq \\
a \geq v & \text{if } \eta = \geq
\end{cases}
\end{equation}


Exactly as in Vanderbeck's branching, we can find such a subset $Q'$ by finding a component bound sequence $S$, such that:
\begin{equation}
\sum_{q \in Q(S)} \lambda_q^* \eqqcolon K \not\in \mathbb{Z}
\end{equation}
where $Q(S) \coloneqq \{q \in Q \mid \forall (x_i, \eta, v) \in S. \eta(x_{qi}, v)\}$.

Using an analogous proof as \ref{pr:cg_bp_bp}, it can be shown that such a $S$ always exists if the master solution is not integral. After we have obtained such a $S$, we can now create two child nodes, the down- and up-branches respectively, by first adding the branching decision to the master problem:
\begin{multicols}{2}
\noindent
\begin{minipage}{\linewidth}
\setlength{\belowdisplayskip}{0pt} \setlength{\belowdisplayshortskip}{0pt}
\setlength{\abovedisplayskip}{0pt} \setlength{\abovedisplayshortskip}{0pt}
\begin{equation*}
\sum_{q \in Q(S)} \lambda_q \leq \lfloor K \rfloor \quad \left[\gamma_{\downarrow} \leq 0\right]
\end{equation*}
\end{minipage}

\columnbreak

\noindent
\begin{minipage}{\linewidth}
\setlength{\belowdisplayskip}{0pt} \setlength{\belowdisplayshortskip}{0pt}
\setlength{\abovedisplayskip}{0pt} \setlength{\abovedisplayshortskip}{0pt}
\begin{equation}
\sum_{q \in Q(S)} \lambda_q \geq \lceil K \rceil \quad \left[\gamma_{\uparrow} \geq 0\right]
\end{equation}
\end{minipage}
\end{multicols}

We now must ensure that newly priced columns $x_{q'}$ are assigned a coefficient of $y = 1$ for the branching decision if $q' \in Q(S)$, i.e. if $\forall (x_i, \eta, v) \in S. \eta(x_{q'i}, v)$ and otherwise $y = 0$. We achieve this by introducing additional binary variables $\bar{y}_s, \ubar{y}_{s'}$ for each $B_s \in \bar{S}$ and for each $B_{s'} \in \ubar{S}$ respectively, along with the following constraints, in the \SP{}:
\begin{equation}
\begin{aligned}
y = 1 &\Leftrightarrow \sum_{B_s \in \bar{S}} \bar{y}_s + \sum_{B_s \in \ubar{S}} \ubar{y}_s = \abs{S} &\\
\bar{y}_s = 1 &\Leftrightarrow x_{i,s} \leq v_s & \forall B_s \in \bar{S} \\
\ubar{y}_s = 1 &\Leftrightarrow x_{i,s} \geq v_s & \forall B_s \in \ubar{S} \\
y &\in \{0, 1\} & \\
\bar{y}_s &\in \{0, 1\} & \forall B_s \in \bar{S} \\
\ubar{y}_s &\in \{0, 1\} & \forall B_s \in \ubar{S}
\end{aligned}
\end{equation}

What remains is expressing all logical equivalences using a finite set of linear constraints. For this, we can use the following observation:
\begin{itemize}
\item Since in the down branch $-\gamma_{\downarrow} \geq 0$, $y$ naturally takes on value $0$ and therefore also all $\bar{y}_s$ and $\ubar{y}_{s'}$. Thus, in the down branch we need to force all $\bar{y}_s$ and $\ubar{y}_{s'}$ to $1$ if the corresponding component bounds are satisfied, and force $y$ to $1$ if all $\bar{y}_s$ and $\ubar{y}_{s'}$ equal $1$.
\item In the up branch, the opposite is the case: since $-\gamma_{\uparrow} \leq 0$, $y$ and all $\bar{y}_s, \ubar{y}_{s'}$ naturally take on value $1$, requiring us to force all $\bar{y}_s$ and $\ubar{y}_{s'}$ to $0$ if their corresponding component bounds are not satisfied, and force $y$ to $0$ if any of the $\bar{y}_s, \ubar{y}_{s'}$ equals $0$.
\end{itemize}

Recall, that we require a bounded \IP{} to begin with. Let us denote the lower and upper bound of a variable $x_i$ as $\text{lb}_i$ and $\text{ub}_i$ respectively. Using the above observations, we can now express the logical equivalences mandated by the branching decision as follows:
\begin{multicols}{2}
\noindent
\begin{minipage}{0.95\linewidth}
\setlength{\belowdisplayskip}{0pt} \setlength{\belowdisplayshortskip}{0pt}
\setlength{\abovedisplayskip}{4pt} \setlength{\abovedisplayshortskip}{4pt}
\begin{flalign*}
y &\geq 1 + \sum_{B_s \in \bar{S}} \bar{y}_s  + \sum_{B_s \in \ubar{S}} \ubar{y}_s - \abs{S} &
\end{flalign*}
\begin{flalign*}
\bar{y}_s &\geq \frac{v_s + 1 - x_{i,s}}{v_s + 1 - \text{lb}_i} &\forall B_s \in \bar{S} \\
\ubar{y}_s &\geq \frac{x_{i,s} - v_s}{\text{ub}_i - v_s} &\forall B_s \in \ubar{S}
\end{flalign*}
\end{minipage}

\columnbreak

\noindent
\begin{minipage}{\linewidth}
\setlength{\belowdisplayskip}{0pt} \setlength{\belowdisplayshortskip}{0pt}
\setlength{\abovedisplayskip}{0pt} \setlength{\abovedisplayshortskip}{0pt}
\begin{equation}
\begin{aligned}
y &\leq \bar{y}_s &\qquad \forall B_s \in \bar{S} \\
y &\leq \ubar{y}_s &\qquad \forall B_s \in \ubar{S} \\
\bar{y}_s &\leq \frac{\text{ub}_i - x_{i,s}}{\text{ub}_i - v_s} &\qquad \forall B_s \in \bar{S} \\
\ubar{y}_s &\leq \frac{x_{i,s} - \text{lb}_i}{v_s - \text{lb}_i} &\qquad \forall B_s \in \ubar{S}
\end{aligned}
\end{equation}
\end{minipage}
\end{multicols}

We have now successfully defined the branching decision in the master problem, and the corresponding constraints in the pricing problem. Until we have found an optimal integral solution of master variables, we will continue to branch using a suitable component bound sequence $S$, creating a binary branch-and-bound search tree. In the next section, we will present an algorithm responsible for finding such a $S$ given a fractional master solution $\vec{\lambda}^*_\RMP{}$.