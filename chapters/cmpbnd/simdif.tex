\section{Comparison to Vanderbeck's Generic Branching}\label{sec:cmpbnd_simdif}
Both the generic branching scheme by Vanderbeck (Section \ref{sec
}) and the proposed component bound scheme involve imposing bounds on the original variables within the \SP{} to branch in the master problem. However, the methods for enforcing these component bounds differ significantly.

Vanderbeck's \texttt{GENERIC} branching scheme treats these bounds as hard constraints, effectively subdividing the solution space of the original variables. As a result, when the optimal solution $\vec{x}^*$ to the \IP{} is found in a node of the \RMP{}, it adheres to all component bounds imposed by the branching decisions from the root to that node. In contrast, our component bound branching rule introduces these bounds as soft constraints, allowing the generation of columns that both satisfy and violate the component bounds.

While the component bound branching rule might be simpler to implement, Vanderbeck's \texttt{GENERIC} branching offers a notable advantage: it only requires tightening the bounds of the original variables in the \SP{}, without introducing new variables and constraints. This simplicity maintains the structure of the pricing problem, enabling many dynamic programming solvers for specific \IP{}s to efficiently generate columns despite changing variable bounds. Conversely, our approach alters the pricing problem with each branching decision, potentially necessitating the use of a generic \MIP{} solver. Moreover, the \texttt{GENERIC} scheme incrementally tightens the bounds as the search tree deepens, making the pricing problem progressively easier to solve. In contrast, \texttt{COMPBND} branching adds more variables and constraints, complicating the \SP{} as the branching process continues.
