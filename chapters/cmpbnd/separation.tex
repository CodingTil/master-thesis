\section{Separation Procedure}\label{sec:cmpbnd_separation}
\begin{definition}
The \textbf{fractionality of} $\vec{\lambda}^*_\RMP{}$ \textbf{ with respect to } $S$ is given by:
\begin{equation}\label{eq:cmpbnd_fractionality}
F_S = \sum_{q \in \mathcal{Q}(S)} \left( \lambda_q^* - \floor{\lambda_q^*} \right) \geq 0
\end{equation}
\end{definition}

When $S = \emptyset$, we have $Q(S) = Q$, and thus $F_S > 0$ since at least one $\lambda_q^*$ is fractional. In this case, $F_S \in \mathbb{Z}_+ \setminus \{0\}$ due to the convexity constraint $\sum_{q \in Q} \lambda_q = 1$ in the \MP{} (analogous in aggregated subproblems, see Section \ref{sec:cg_bp_idsp}).

In general, for any $S$ one of three cases can occur:
\begin{itemize}
\item	$F_S = 0$: $Q(S)$ contains no column with fractional $\lambda_q^*$. Thus, branching on $S$ would not cut off the current fractional solution $\vec{\lambda}^*_\RMP{}$. Adding further component bounds to $S$ would not change this.
\item	$a < F_S < a + 1, a \in \mathbb{Z}_+$. Using Equation \eqref{eq:cmpbnd_fractionality}, we can rewrite this as:
		\begin{equation}
		\sum_{q \in \mathcal{Q}(S)} \floor{\lambda_q^*} < \sum_{q \in \mathcal{Q}(S)} \lambda_q^* < \sum_{q \in \mathcal{Q}(S)} \floor{\lambda_q^*} + 1
		\end{equation}
		The sum $\sum_{q \in \mathcal{Q}(S)} \lambda_q^* \eqcolon K$ is fractional, enabling us to branch on $S$ (Equation \eqref{eq:compbnd_branching_master}).
\item	$F_S \in \mathbb{Z}_+ \setminus \{0\}$. In this case, $\sum_{q \in \mathcal{Q}(S)} \lambda_q^* \in \mathbb{Z}_+$, and therefore branching on $S$ would not cut off the current fractional solution. However, using \ref{note:distinct_columns}, we can find two distinct columns $q_1, q_2 \in Q(S)$, i.e., where $x_{i,q_1} < x_{i,q_2}$ for some $i \in \{1, \dots, n\}$, such that $\lambda_{q_1}^*$ and $\lambda_{q_2}^*$ are fractional. Denote the rounded median of these two column entries as $v \coloneqq \floor{\frac{x_{i,q_1} + x_{i,q_2}}{2}}$. Since $x_{i,q_1} \leq v < v + 1 \leq x_{i,q_2}$, we can separate $q_1$ from $q_2$ by imposing a bound on the component $x_i$, i.e., expand $S$ to either $S_1$ or $S_2$, where:
		\begin{equation}
		\begin{aligned}
		S_1 &\coloneqq S \cup \{\left( x_i, \leq, v \right)\}\\
		S_2 &\coloneqq S \cup \{\left( x_i, \geq, v + 1 \right)\}
		\end{aligned}
		\end{equation}
		Note that $F_S = F_{S_1} + F_{S_2}$, thus we can always at least halve the fractionality of the current solution. Furthermore, both $Q(S_1)$ and $Q(S_2)$ are guaranteed to contain at least one fractional column, ensuring $F_{S_1}, F_{S_2} > 0$.
\end{itemize}

These observations suggest the following separation procedure: initialize $S^0 = \emptyset$, i.e., $F_{S^0} > 0$. While $F_{S^k} \in \mathbb{Z}_+ \setminus \{0\}$, find a component bound $x_i$ to branch on, yielding $S_1$ and $S_2$. Proceed with either as $S^{k+1}$. Finally, $F_{S^k}$ will be fractional, and we can branch on $S^k$ \cite{thebook}.

\begin{proposition}
At no iteration $k \geq 0$ will the separation procedure produce a component bound sequence $S^k$ with $F_{S^k} = 0$.
\end{proposition}

\begin{proof}
As previously discussed, $F_\emptyset > 0$, i.e., $S^0$ satisfies the proposition.

Assume $S^k$ satisfies the proposition, i.e., $F_{S^k} > 0$. If $F_{S^k} \not\in \mathbb{Z}_+$, the procedure terminates, and the proposition holds. Else $F_{S^k} \not\in \mathbb{Z}_+ \setminus \{0\}$. In this case, let us assume $F_{S^{k+1}} = 0$. Then $Q(S^{k+1})$ contains no fractional columns, which contradicts the design of $S^{k+1}$. By contradiction, $F_{S^{k+1}} > 0$ must hold, and by induction, the proposition holds.
\end{proof}

\begin{proposition}
Given that $\vec{\lambda}^*_\RMP{}$ contains finitely many non-zero values, the separation procedure will terminate after a finite number of iterations.
\end{proposition}

\begin{proof}
Let us denote the restriction of $Q(S)$ to the columns $q$ with fractional $\lambda_q^*$ as $Q_f(S)$. By our assumption $\abs{Q_f(S)} < \infty$. At each iteration $k$, we only remove columns from $Q_f(S^k)$, i.e., $\abs{Q_f(S^{k+1})} < \abs{Q_f(S^k)}$. Since $\abs{Q_f(S^0)} < \infty$, the separation procedure must terminate after a finite number of iterations.
\end{proof}

\subsection{Choice of Component Bounds}\label{sec:cmpbnd_separation_choice}
The separation procedure described above is not complete, as we have not yet defined which bounds we impose on which components. This choice can significantly impact the performance of the subsequent solving of the child nodes. In the worst case, the separation procedure will yield a component bound sequence $S$ for which $Q(S)$ only contains one column, i.e., dichotomous branching. Maintaining balance within the tree is generally beneficial, but the time required to find an optimal $S$ can grow arbitrarily large and must be traded off against improved performance that comes with a balanced tree. We propose the following two-staged approach:

In the first stage, using one or multiple heuristics, we recursively determine a set of valid component bound sequences $S_1, \dots, S_m$ for the current fractional master solution $\vec{\lambda}^*_\RMP{}$. For this, we adapt the previously described separation procedure to explore both options $S_1$ and $S_2$ whenever $F_S$ is integral. A first-stage heuristic is now only responsible for finding a separating component $x_i$ and bound value $v \in \mathbb{Z}$. Both the lower bound $x_i \leq v$ and the upper bound $x_i \geq v + 1$ will be explored further; we do not have to choose between them at this stage. In particular, we propose two heuristics for this first stage:

\begin{itemize}
\item	\texttt{MaxRangeMidrange} Heuristic: At each iteration $k$, we choose the component $x_i$ for which the components $x_{i,q}$ of the columns $q \in Q_f(S^k)$ are most spread out. We then bound $x_i$ by the midrange of these components. Formally, we define:
		\begin{equation*}
		\begin{aligned}
		max_j &\coloneqq \underset{q \in Q_f(S^k)}{\arg\max} \; x_{j,q} & \forall j \in \{1, \dots, n\}\\
		min_j &\coloneqq \underset{q \in Q_f(S^k)}{\arg\min} \; x_{j,q} & \forall j \in \{1, \dots, n\}\\
		x_i &= \underset{j \in \{1, \dots, n\}}{\arg\max} \; max_j - min_j & \\
		v &\coloneqq \frac{max_i - min_i}{2} &
		\end{aligned}
		\end{equation*}
\item	\texttt{MostDistinctMedian} Heuristic: At each iteration $k$, we choose the component $x_i$ for which the components $x_{i,q}$ of the columns $q \in Q_f(S^k)$ have the most distinct values. We then choose $v$ to be the median of these components.
\end{itemize}

In the second stage, another heuristic now scores every component bound sequence, and we continue branching using the highest scoring $S_j$. Specifically, we propose to choose the smallest component bound sequence, i.e., $S_j = \arg\min_{S_1, \dots, S_m} \abs{S_j}$. This minimizes the modifications we make to the pricing problem. In case there are multiple such minimal component bound sequences, we propose to use one of two further heuristics to select one component bound sequence out of all those with minimal cardinality:

\begin{itemize}
\item	\texttt{ClosedToZHalf} Heuristic: For each $S_j$, we calculate $K_j \coloneqq \sum_{q \in Q(S_j)} \lambda_q^*$. We then select the one where $K_j$ is closest to $\frac{Z}{2}$. Here, $Z$ denotes the number of aggregated subproblems in the current block (Section \ref{sec:cg_bp_idsp} and Section \ref{sec:cmpbnd_separation_branching}). The intuition behind this is once again to maintain balance within the tree: if $K_j$ was far off from $\frac{Z}{2}$, i.e., either close to $0$ or close to $Z$, branching on $S_j$ would be similar to dichotomous branching, since it would either forbid almost all or almost no solutions (Section \ref{sec:cg_bp_bp_branching_master}).
\item	\texttt{MostFractional} Heuristic: Here, we also calculate $K_j$ for each $S_j$. We then select the one where $K_j$ is most fractional, i.e., where $K_j - \floor{K_j}$ is closest to $0.5$. This heuristic is motivated by the idea that such a most fractional selection of master variables can be interpreted as the \RMP{} being most uncertain about.
\end{itemize}

\subsection{Post-processing of Component Bound Sequences}\label{sec:cmpbnd_separation_postprocessing}
Depending on the heuristics chosen, there is no guarantee that the separation procedure will find a component bound sequence $S$ in which each component $x_i$ has at most one upper bound (lower bound analogous). While this is not a problem from a mathematical standpoint, only the least upper bound (greatest lower bound, respectively) is relevant, and so adding variables and constraints for the other upper bounds (lower bounds) is unnecessary and could potentially slow down the solving process of \SP{}. Therefore, post-processing of the component bound sequences, i.e., removing redundant bounds, is advisable. For example, if we had $S = \{(x_1, \leq, 3), (x_1, \leq, 4)\}$, the first bound already implies the second, and we can remove the second bound from $S$, yielding $\{(x_1, \leq, 3)\}$.

\subsection{Branching with Multiple Subproblems}\label{sec:cmpbnd_separation_branching}
The component bound branching rule described above can be applied to instances with a single subproblem, as well as instances with multiple identical subproblems aggregated into a single subproblem (Section \ref{sec:cg_bp_idsp}). However, there are instances consisting of at least two distinct subproblems, also known as blocks, where the master problem yields a solution ${\vec{\lambda}^k}^*_\RMP{}$ for each block $k$. Since each component $x_i$ belongs to a specific block, not all columns $q_1, q_2$ in \RMP{} will have an entry for $x_i$, thus the separation scheme is not directly applicable across multiple blocks.

Given that more than one block has fractional master solutions, we propose to pick one of those blocks to branch on and then apply the separation procedure as described above within the selected block.
