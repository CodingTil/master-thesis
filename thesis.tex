\documentclass[
	12pt, % Font size
	a4paper, % Paper size
	twoside, % Two-sided
]{report}

% Set the PDF version for LuaTeX
\usepackage{luacode}
\begin{luacode}
  pdf.setminorversion(6)
\end{luacode}

% Packages
\usepackage[english]{babel} % Set document language
\usepackage{pdfpages} % Include PDF files
\usepackage[utf8]{inputenc} % UTF-8 encoding (for umlauts etc.)
\usepackage[T1]{fontenc} % correct hyphenation
\usepackage{csquotes} % correct quotation marks
\usepackage{lmodern} % Computer Modern fonts
\usepackage{microtype} % better typesetting results (avoids underfull / overfull hboxes)
\usepackage{graphicx} % adding graphics
\usepackage{units} % typesetting units, e.g. \unit[10]{MB} and \unitfrac[100]{Mbit}{s}
\usepackage{booktabs} % publication quality tables
\usepackage{titlesec} % Customize chapter and section headings
\usepackage{setspace} % Set line spacing
\usepackage{geometry} % Set page margins
\usepackage{subcaption} % Subfigures
\usepackage{float} % Floats
\usepackage[
	backend=bibtex,
	style=numeric-comp,
	maxcitenames=2,
	natbib=true,
	sorting=none
]{biblatex} % Bibliography
\usepackage{hyperref} % Hyperlinks
\usepackage[super]{nth} % Superscript for numbers
\usepackage{amsmath} % Math symbols
\usepackage{amssymb} % Math symbols
\usepackage{mathtools} % Math symbols
\usepackage{amsthm} % Math symbols
\usepackage[
	ruled,
	linesnumbered,
	algochapter
]{algorithm2e} % Algorithms
\usepackage{bm} % Bold math symbols
\usepackage{tikz} % Drawing
\usepackage{tikz-qtree} % Drawing trees
\usepackage{etoolbox} % Patching
\usepackage{multicol} % Multiple columns
\usepackage{accents} % Math accentsdef

\setlength{\columnseprule}{1pt}

\setcounter{secnumdepth}{3}
\setcounter{tocdepth}{2}

% Bibliography
\addbibresource{literature.bib}

% Custom Definitions
% Theorems %
\newtheorem{theorem}{Theorem}
\numberwithin{theorem}{chapter}
\newtheorem{definition}{Definition}
\numberwithin{definition}{chapter}
\newtheorem{corollary}{Corollary}
\numberwithin{corollary}{chapter}

% Algorithm2e Keywords %
\SetKwBlock{Loop}{loop}{}
\SetKw{Break}{break}
\SetKw{None}{None}

% Functions &
\newcommand{\polyhedron}[1]{\mathcal{#1}}
\newcommand{\mat}[1]{{\bm{#1}}}
\renewcommand{\vec}[1]{{\bm{#1}}}
\newcommand{\conv}[0]{\operatorname{conv}}
\newcommand{\rayspan}[0]{\operatorname{rayspan}}
\newcommand{\transpose}[0]{^\intercal}
\renewcommand{\setminus}{\backslash}
\newcommand{\rank}[0]{\operatorname{rank}}
\newcommand{\st}[0]{\operatorname{s.t.}}
\newcommand{\MP}[0]{\textit{MP}}
\newcommand{\RMP}[0]{\textit{RMP}}
\newcommand{\SP}[0]{\textit{SP}}
\newcommand{\indexset}[1]{\mathcal{#1}}
\DeclarePairedDelimiter{\abs}{\lvert}{\rvert}


% Document
\begin{document}

\begin{titlepage}
	\centering
	{\Huge\bfseries Component Bound Branching in a Branch-and-Price Framework\par}
	\vspace{0.85cm}
	{\LARGE Master Thesis in Computer Science\\RWTH Aachen University\par}
	\vspace{2cm}
	{\LARGE Til Mohr\par}
	\vspace{0.5cm}
	{\large til.mohr@rwth-aachen.de\\Student ID: 405959\par}
	\vspace{2cm}
	{\large \today\par}
	\vspace{2cm}
	\begin{minipage}{0.48\textwidth}
		\centering
		\nth{1} Examiner\\
		Prof. Dr. Peter Rossmanith\\
		Chair of Theoretical Computer Science\\
		RWTH Aachen University
	\end{minipage}
	\begin{minipage}{0.48\textwidth}
		\centering
		\nth{2} Examiner\\
		Prof. Dr. Marco Lübbecke\\
		Chair of Operations Research\\
		RWTH Aachen University
	\end{minipage}
\end{titlepage}

% Eidesstattliche Versicherung
\includepdf{Formular_Eidesstattliche_Versicherung.pdf}

% Abstract
\begin{abstract}
This master thesis integrates the component bound branching rule, proposed by Vanderbeck et al. \cite{vanderbeck2010reformulation,vanderbeck1996exact}, into the branch-price-and-cut solver GCG. This rule, similarly to Vanderbeck's generic branching scheme \cite{vanderbeck2011branching}, exclusively operates within the Dantzig-Wolfe reformulated problem, where branching decisions generally have no corresponding actions in the original formulation. The current GCG framework requires modifications for such branching rules, especially within the pricing loop, as seen in Vanderbeck's method implementation. These rules also fail to utilize enhancements like dual value stabilization.

A significant contribution of this thesis is the enhancement of the GCG architecture to facilitate the seamless integration of new branching rules that operate solely on the reformulated problem. This allows these rules to benefit from current and future improvements in the branch-price-and-cut framework, including dual value stabilization, without necessitating alterations to the branching rule itself.

The thesis proposes an interface to manage constraints in the master problem that lack counterparts in the original formulation. These constraints require specific modifications to the pricing problems to ensure their validity in the master. The 'generic mastercut' interface, tightly integrated into the GCG solver, spans the pricing loop, column generation, and dual value stabilization. Enhancements to the existing branching rule interface complement this integration, enabling effective utilization of the generic mastercuts.

Finally, the component bound branching rule will be implemented using this new interface and evaluated on a set of benchmark instances. Its performance will be benchmarked against the existing Vanderbeck branching rule, offering a practical comparison of both approaches.
\end{abstract}
\cleardoublepage

% Table of contents
\tableofcontents
\cleardoublepage

% Chapters
\chapter{Introduction}
The development of efficient algorithms for solving large-scale mixed-integer programming (\MIP{}) problems has been a central focus of operations research for decades. Column generation, a powerful technique for solving large-scale linear programs, has been extended to integer programs through the branch-and-price algorithm. The effectiveness of branch-and-price relies heavily on the branching strategies employed and the ability to integrate various constraints into the master problem during column generation. However, these techniques introduce complexities that require careful management to ensure computational efficiency.

In this thesis, we explore advanced branching rules and constraints within the context of the branch-and-price framework, particularly focusing on the implementation and evaluation of the component bound branching rule. This rule, as proposed by Desrosiers et al. \cite{thebook}, offers a simpler alternative to Vanderbeck's generic branching scheme \cite{vanderbeck2011branching} by allowing for more flexible management of branching decisions within the master problem. The component bound branching rule provides an approach that is potentially easier to implement and integrate with modern solvers such as \GCG{}, yet it introduces challenges that necessitate modifications to the solver's architecture.

The foundational concepts of polyhedron representation and the primal simplex algorithm are introduced in Chapter \ref{ch:preliminaries}, providing the mathematical and algorithmic background necessary for understanding the core methods discussed later. In Chapter \ref{ch:cg_bp}, we delve into the specifics of column generation and branch-and-price, detailing the algorithms and their implementation, including the Dantzig-Wolfe reformulation, which serves as the basis for the decomposition approach used in branch-and-price.

Chapter \ref{ch:tools} provides a brief overview of the \SCIP{} Optimization Suite, with a particular focus on the \GCG{} solver, which forms the foundation for the implementation work carried out in this thesis. In Chapter \ref{ch:cmpbnd}, we present the component bound branching rule in detail, including its separation procedure and a theoretical comparison to Vanderbeck's generic branching scheme. This analysis sets the stage for a broader discussion on the limitations and advantages of different branching strategies in column generation.

One of the major contributions of this thesis is the introduction of a new interface within \GCG{} for handling constraints that exist solely within the master problem, termed \texttt{generic mastercuts}. These constraints, which do not have a direct counterpart in the original problem, require special handling within the solver, particularly with respect to synchronizing master variables across the search tree and applying dual value stabilization. Chapter \ref{ch:gm} begins by presenting the conceptual framework and definition of generic mastercuts, followed by an elaboration on the synchronization mechanisms and the application of dual value stabilization for these constraints, highlighting the technical innovations introduced in this thesis.

Chapter \ref{ch:implementation} focuses on the implementation aspects, detailing how the generic mastercut interface was integrated into \GCG{}, and how it supports the component bound branching rule and other advanced branching techniques.

The effectiveness of the component bound branching rule and the generic mastercut interface is rigorously evaluated in Chapter \ref{ch:evaluation}. We compare different separation heuristics and analyze the impact of dual value stabilization on the performance of the branching rule. Additionally, a detailed comparison with Vanderbeck's generic branching scheme provides insights into the conditions under which the component bound branching rule may offer advantages.

Finally, in Chapter \ref{ch:conclusion}, we conclude by summarizing the contributions of this thesis, reflecting on the broader implications of our findings for the field of operations research, and suggesting directions for future research. This thesis not only contributes to the theoretical understanding of advanced branching strategies but also provides practical tools for their implementation in state-of-the-art solvers, paving the way for more efficient solutions to complex integer programming problems.

\cleardoublepage
\chapter{Preliminaries}\label{ch:preliminaries}

This preliminary chapter serves the purposes of introducing fundamental notation that will be used throughout this thesis. In doing so, we will provide a brief rundown of theorems and algorithms on which the techniques described in later chapters are building upon. For a more detailed introduction to the topics covered in this chapter, we refer to Chapter 1 of the book \textit{Branch-and-Price} by Desrosiers et al. \cite{thebook}.

\section{Polyhedron Representation}\label{sec:preliminaries_poly}

% Conic
\begin{definition}
Given $k$ points $\vec{x}_1, \dots, \vec{x}_k \in \mathbb{R}^n$, any $\vec{x} = \sum_{i=1}^{k} \alpha_i \vec{x}_i$ is a \textbf{conic combination} of the $\vec{x}_i$, iff $\forall i \in \{1, \dots, k\}. \alpha_i \geq 0$.
\end{definition}

% Convexity
\begin{definition}\label{def:convex}
Given $k$ points $\vec{x}_1, \dots, \vec{x}_k \in \mathbb{R}^n$, any $\vec{x} = \sum_{i=1}^{k} \alpha_i \vec{x}_i$ is a \textbf{convex combination} of the $\vec{x}_i$, iff $\sum_{i=1}^{k} \alpha_i = 1 \land \forall i \in \{1, \dots, k\}. \alpha_i \geq 0$.

The set of all convex combinations of $\vec{x}_1, \dots, \vec{x}_k$ is therefore defined as:
\begin{equation*}
\conv(\vec{x}_1, \dots, \vec{x}_k) \coloneqq \{\sum_{i=1}^{k} \alpha_i \vec{x}_i \mid \sum_{i=1}^{k} \alpha_i = 1 \land \forall i \in \{1, \dots, k\}. \alpha_i \geq 0\}
\end{equation*}
\end{definition}

\begin{corollary}\label{cor:intersection_convex}
The intersection of two convex sets is convex.
\end{corollary}

% Extreme Points
\begin{definition}
Let $\polyhedron{P}$ be a convex set. A point $\vec{p} \in \polyhedron{P}$ is an \textbf{extreme point} of $\polyhedron{P}$ if there is no non-trivial convex combination of any two points in $\polyhedron{P}$ expressing $\vec{p}$, i.e.
\begin{equation*}
\forall \vec{x}_1, \vec{x}_2 \in \polyhedron{P}. \forall \alpha \in \mathbb{R}_+ \setminus \{0\}. \vec{x}_1 \neq \vec{x}_2 \implies \vec{p} \neq \alpha \vec{x}_1 + (1 - \alpha) \vec{x}_2
\end{equation*}
\end{definition}

% Rays
\begin{definition}\label{def:rays}
Let $\polyhedron{P}$ be a convex set. A vector $\vec{r} \in \mathbb{R}_0^n \setminus \{0\}$ is a \textbf{ray} of $\polyhedron{P}$ iff $\forall \vec{x} \in \polyhedron{P}. \forall \beta \in \mathbb{R}_+. \vec{x} + \beta \vec{r} \in \polyhedron{P}$.

The span of rays $\vec{r}_1, \dots, \vec{r}_k \in \mathbb{R}_+^n$ we denote as:
\begin{equation*}
\rayspan(\vec{r}_1, \dots, \vec{r}_k) \coloneqq \bigcup_{i=1}^{k} \{\omega \vec{r}_i \mid \omega \in \mathbb{R}_+\}
\end{equation*}
\end{definition}

\begin{definition}
A ray $\vec{r}$ of $\polyhedron{P}$ is an \textbf{extreme ray} of $\polyhedron{P}$ if there is no non-trivial conic combination of any two rays in $\polyhedron{P}$ expressing $\vec{r}$, i.e.
\begin{equation*}
\forall \vec{r}_1, \vec{r}_2 \in \polyhedron{P}. \forall \alpha_1, \alpha_2, \beta \in \mathbb{R}_+ \setminus \{0\}. \vec{r}_1 \neq \beta \vec{r}_2 \implies \vec{r} \neq \alpha_1 \vec{r}_1 + \alpha_2 \vec{r}_2
\end{equation*}
\end{definition}

% Hyperplane
\begin{definition}
A \textbf{hyperplane} $\polyhedron{H} \subset \mathbb{R}^n$ of a $n$-dimensional space is a subspace of dimension $n-1$, and can therefore be described using a vector $\vec{f} \in \mathbb{R}^n$ and a scalar $f \in \mathbb{R}$ as $\polyhedron{H} = \{\vec{x} \mid \vec{f}\transpose \vec{x} = f\}$.
\end{definition}

\begin{corollary}
Any hyperplane is a convex set.
\end{corollary}

\begin{proof}
Let $\polyhedron{H} = \{\vec{x} \mid \vec{f}\transpose \vec{x} = f\}$ be a hyperplane. Let $k \in \mathbb{N}$, $\vec{x}_1, \dots, \vec{x}_k \in \polyhedron{H}$. For any $\alpha_1, \dots, \alpha_k \in \mathbb{R}_+$ with $\sum_{i=1}^{k}$:
\begin{align*}
\vec{f}\transpose \left( \sum_{i=1}^{k} \alpha_i \vec{x}_i \right)
&= \sum_{i=1}^{k} \alpha_i \vec{f}\transpose \vec{x}_i \\
&= \sum_{i=1}^{k} \alpha_i \cdot f \\
&= f \cdot \sum_{i=1}^{k} \alpha_i \\
&= f
\end{align*}
Therefore, the convex combination $\sum_{i=1}^{k} \alpha_i \vec{x}_i$ is in the hyperplane $\polyhedron{H}$.
\end{proof}

% Halfspace
\begin{definition}
A \textbf{halfspace} is the set above or below a hyperplane. A halfspace is open if the points on the hyperplane are excluded, otherwise closed.
\end{definition}

\begin{corollary}\label{cor:halfspace_convex}
Any halfspace is a convex set.
\end{corollary}

\begin{proof}
Let $\polyhedron{H}^+ = \{\vec{x} \mid \vec{f}\transpose \vec{x} > f\}$ be an open halfspace (analogous for $\polyhedron{H}^- = \{\vec{x} \mid \vec{f}\transpose \vec{x} < f\}$, and for the closed halfspaces). Let $k \in \mathbb{N}$, $\vec{x_1}, \dots, \vec{x_k} \in \polyhedron{H}$. For any $\alpha_1, \dots, \alpha_k \in \mathbb{R}_+$ with $\sum_{i=1}^{k}$:
\begin{align*}
\vec{f}\transpose \left( \sum_{i=1}^{k} \alpha_i \vec{x}_i \right)
&= \sum_{i=1}^{k} \alpha_i \vec{f}\transpose \vec{x}_i \\
&> \sum_{i=1}^{k} \alpha_i \cdot f \\
&= f \cdot \sum_{i=1}^{k} \alpha_i \\
&= f
\end{align*}
Therefore, the convex combination $\sum_{i=1}^{k} \alpha_i \vec{x}_i$ is in the halfspace $\polyhedron{H}$.
\end{proof}

% Polyhedron
\begin{definition}
A \textbf{polyhedron} $\polyhedron{P} \subseteq \mathbb{R}^n$ is defined by the intersection of a set of closed halfspaces, i.e. $\polyhedron{P} \coloneqq \{\vec{x} \in \mathbb{R}^n \mid \mat{A} \vec{x} \geq \vec{b}\}$, with $\mat{A} \in \mathbb{R}^{m \times n}, \vec{b} \in \mathbb{R}^m$.

By Corollaries \ref{cor:intersection_convex} and \ref{cor:halfspace_convex}, a polyhedron is also a convex set of points.
\end{definition}

% Ray

% Monkowski-Weyl
\begin{definition}
The \textbf{Minkowski sum} of two sets $P, Q$ is defined by:
\begin{equation*}
P \oplus Q \coloneqq \{\vec{p} + \vec{q} \mid \vec{p} \in P \land \vec{q} \in Q\}
\end{equation*}
\end{definition}

\begin{theorem}[Minkowski-Weyl]\label{th:minkowski-weyl}
For $\polyhedron{P} \subseteq \mathbb{R}^n$ the following statements are equivalent:
\begin{enumerate}
\item $\polyhedron{P}$ is a polyhedron, i.e., there exists some finite matrix $\mat{A} \in \mathbb{R}^{m \times n}$ and some vector $\vec{b} \in \mathbb{R}^m$ such that $P = \{\vec{x} \in \mathbb{R}^n \mid \mat{A} \vec{x} \leq \vec{b}\}$
\item There exist fine vectors $\vec{v}_1, \dots, \vec{v}_s \in \mathbb{R}^n$ and finite vectors $\vec{r}_1, \dots, \vec{r}_t \in \mathbb{R}_+^n$, such that $P = \conv(\vec{v}_1, \dots, \vec{v}_s) \oplus \rayspan(\vec{r}_1, \dots, \vec{r}_t)$
\end{enumerate}
\end{theorem}

In simple terms, the Minkowski-Weyl theorem states that any polyhedron can always be defined in two ways: either by its faces, i.e. closed halfspaces, or by its vertices and rays. Because of their unique properties, for such representation of a polyhedron it is sufficient to select its the extreme points and extreme rays. Figure xyz. illustrates this TODO-til

The following theorem builds upon the Minkowski-Weyl theorem to describe a polyhedron, which is represented by its extreme points $\{\vec{x}_p\}_{p \in P}$ and extreme rays $\{\vec{x}_r\}_{r \in R}$, using hyperplanes. Here, the sets $P, R$ are used to index the extreme points and extreme rays, respectively.

\begin{theorem}[Nemhauser-Wolsey]\label{th:nemhauser-wolsey}
Consider the polyhedron $\polyhedron{P} = \{\vec{x} \in \mathbb{R}^n \mid \mat{Q} \vec{x} \geq \vec{b}\}$ with full row rank matrix $\mat{Q} \in \mathbb{R}^{m \times n}$, i.e. $\rank(\mat{Q}) = m \leq n \land \polyhedron{P} \neq \emptyset$.
An equivalent description of $\polyhedron{P}$ using its extreme points $\{\vec{x}_p\}_{p \in P}$ and extreme rays $\{\vec{x}_r\}_{r \in R}$ is:
\begin{equation}
\polyhedron{P} = \left\{ \vec{x} \in \mathbb{R}^n \middle\vert
\begin{aligned}
\sum_{p \in P} \vec{x}_p \lambda_p &+ &\sum_{r \in R} \vec{x}_r \lambda_r &= \vec{x} &\\
\sum_{p \in P} \lambda_p & & &= 1 &\\
\lambda_p & & &\geq 0 &\forall p \in P\\
& &\lambda_r &\geq 0 &\forall r \in R
\end{aligned}
\right\}
\end{equation}
\end{theorem}

In the Nemhauser-Wolsey theorem, the conditions of the Minkowski-Weyl theorem are clearly encoded: the second and third lines ensure that the convex set of the extreme points are considered in the first line (Definition \ref{def:convex}), the last playing a part in the span of extreme rays (Definition \ref{def:rays}), and the first line being the Minkowski sum of the convex hull of extreme rays and the span of extreme rays.

\section{Primal Simplex Algorithm}\label{sec:preliminaries_psa}
Have the following linear program in standard form:
\begin{equation}\label{eq:lp_standard}
\begin{aligned}
&\min & \vec{c}\transpose \vec{x} & & \\
&\st & \mat{A} \vec{x} & = \vec{b} & \left[\vec{\pi}\right] \\
&& \vec{x} & \geq \vec{0} &
\end{aligned}
\end{equation}

The primal simplex algorithm finds an optimal solution by moving from one extreme point of the polyhedron to the next, therefore always remaining feasible. A central part of this algorithm is the sufficient optimality condition. For a basic solution $\vec{X} = \left[\vec{x}_\indexset{B}, \vec{x}_\indexset{N}\right]$ at a given extreme point to be optimal, the reduced costs $\bar{c}_j \coloneqq c_j - \vec{\pi}\transpose \vec{a}_j$ for $j \in \indexset{N}$ must be non-negative.

This sufficient optimality condition gives rise to the \textbf{pricing problem}, which either verifies the optimality of the current basic solution, and otherwise determines the non-basic variable $x_l$, $l \in \indexset{N}$ with the least reduced cost ($\bar{c}_l < 0$) to be swapped into the basis next, according to Dantzig's rule (TODO cite). Formally, this can be written as:
\begin{equation}
l \in \underset{j \in \indexset{N}}{\arg\min} \, c_j - \vec{\pi}\transpose \vec{a}_j
\end{equation}
or as the linear program:
\begin{equation}\label{eq:psa_pp}
\bar{c}(\vec{\pi}) = \underset{j \in \indexset{N}}{\min} \, c_j - \vec{\pi}\transpose \vec{a}_j
\end{equation}

Solving the pricing problem thus plays an integral role in the primal simplex algorithm:

\begin{algorithm}
\caption{Primal simplex algorithm with Dantzig's rule}
\KwIn{\LP{} in standard form (\ref{eq:lp_standard}); Basic and non-basic index-sets $\indexset{B},\indexset{N}$}
\KwOut{Optimal Solution $(\vec{x}, z)$}
\Loop{
	$\vec{\pi}\transpose \gets \vec{c}_\indexset{B}\transpose \mat{A}_\indexset{B}^{-1}$;
	$\bar{\vec{b}} \gets \mat{A}_\indexset{B}^{-1} \vec{b}$\;
	$\bar{c}_j \gets c_j - \vec{\pi}\transpose \vec{a}_j$;$ \qquad \forall j \in \indexset{N}$\\
	$l \gets \underset{j \in \indexset{N}}{\arg\min} \, \bar{c}_j$;
	$\bar{c}(\vec{\pi}) \gets \bar{c}_l$\;
	\If{$\bar{c}(\vec{\pi}) \geq 0$}{
		\Return{$\left(\left[\bar{\vec{b}}, \vec{0}\right], \vec{c}_\indexset{B}\transpose \vec{x}_\indexset{B}\right)$} \textit{by optimality}
	}
	$\bar{\vec{a}}_l \gets \mat{A}_\indexset{B}^{-1} \vec{a}_l$\;
	\If{$\bar{\vec{a}}_l \leq \vec{0}$}{
		\Return{\None} \textit{by unboundedness}
	}
	$s \gets \underset{i \in \{1, \dots, m\}}{\arg\min} \, \frac{\bar{b}_i}{\bar{a}_{il}}$;
	$x_l \gets \frac{\bar{b}_s}{\bar{a}_{sl}}$;
	$\indexset{B} \gets \indexset{B} \cup \{l\} \subseteq \{s\}$;
	$\indexset{N} \gets \indexset{N} \cup \{s\} \subseteq \{l\}$\;
}
\end{algorithm}

\cleardoublepage
\chapter{Column Generation and Branch-and-Price}\label{ch:cg_bp}

\section{Column Generation}\label{sec:cg_bp_cg}
Let us consider the following linear program, which we will henceforth call the \textbf{master problem} \MP{}, where $c_\vec{x} \in \mathbb{R}, \vec{a}_\vec{x}, \vec{b} \in \mathbb{R}^m, \forall \vec{x} \in \indexset{X}$:
\begin{equation}
\begin{aligned}
z_\MP{} = &\min & \sum_{\vec{x} \in \indexset{X}} c_\vec{x} \lambda_\vec{x} & & & \\
&\st & \sum_{\vec{x} \in \indexset{X}} \vec{a}_\vec{x} \lambda_\vec{x} & \geq \vec{b} & \left[\vec{\pi}\right] & \\
&& \lambda_\vec{x} & \geq \vec{0} & & \forall \vec{x} \in \indexset{X}
\end{aligned}
\end{equation}
Assume the number of variables is huge, i.e. a lot larger than the number of constraints ($m \ll \abs{\indexset{X}} < \infty$). Because of this, solving \MP{} in a reasonable amount of time, sometimes at all, is infeasible.

We can, however, make use of a crucial property of the primal simplex algorithm: at any given vertex solution, only few variables are in the basis. Most variables are in the non-basis, and therefore have a solution value of $0$. Having a solution value of $0$ is equivalent to not being in the linear program at all. Therefore, the primal simplex algorithm can also function using a manageable subset of variables $\indexset{X}' \subseteq \indexset{X}$, finding a possibly non-optimal, yet still feasible solution for the entire optimization problem \MP{}. We denote this master problem restricted to a subset of variables as the \textbf{restricted master problem} \RMP{}:
\begin{equation}
\begin{aligned}
z_\RMP{} = &\min & \sum_{\vec{x} \in \indexset{X}'} c_\vec{x} \lambda_\vec{x} & & & \\
&\st & \sum_{\vec{x} \in \indexset{X}'} \vec{a}_\vec{x} \lambda_\vec{x} & \geq \vec{b} & \left[\vec{\pi}\right] & \\
&& \lambda_\vec{x} & \geq \vec{0} & & \forall \vec{x} \in \indexset{X}'
\end{aligned}
\end{equation}

Assuming \MP{} is feasible, two important aspects of finding an optimal solution to \MP{} are still missing: first, how do we find a subset $\indexset{X}'$ of the variables, such that \RMP{} stays feasible? Without this property of the set of variables, no solution of \RMP{} can be found, and therefore none can be found for \MP{}, which would contradict the feasibility of \MP{}. Secondly, assuming a solution of \RMP{} was found, possibly even optimal within \RMP{}, how could we build upon this solution to eventually find an optimal solution for \MP{}?

In the following we will dive into these two questions in detail (Sections \ref{sec:cg_bp_cg_farkas} and \ref{sec:cg_bp_cg_reduced}), making way for the final column generation algorithm (Section \ref{sec:cg_bp_cg_alg}).

\subsection{Farkas Pricing}\label{sec:cg_bp_cg_farkas}
Let us assume \MP{} is feasible, but our current selection of variables $\indexset{X}' \subset \indexset{X}$ results in the \RMP{} being infeasible. The task is now to find additional variables such that a new set $\indexset{X}''$ with $\indexset{X}' \subset \indexset{X}'' \subseteq \indexset{X}$ makes the \RMP{} feasible. For this, consider Farkas' lemma:

\begin{theorem}[Farkas' lemma]\label{th:farkas_lemma}
Given $\mat{A} \in \mathbb{R}^{m \times n}$ and $\vec{b} \in \mathbb{R}^m$, then exactly one of the following statements holds:
\begin{enumerate}
	\item $\exists \vec{x} \in \mathbb{R}_+^n. \, \mat{A} \vec{x} \geq \vec{b}$
	\item $\exists \vec{\pi} \in \mathbb{R}_+^n. \, \vec{\pi}\transpose \mat{A} \leq \vec{0} \land \vec{\pi}\transpose \vec{b} > 0$
\end{enumerate}
\end{theorem}

Given that the \MP{} is feasible, the following must hold for the \MP{} with $\mat{A} = \mat{A}_{\vert \indexset{X}}$:
\begin{equation}
\begin{aligned}
& \exists \vec{x} \in \mathbb{R}_+^n. \, \mat{A} \vec{x} \geq \vec{b} \qquad \land \neg \exists \vec{\pi} \in \mathbb{R}_+^n. \, \vec{\pi}\transpose \mat{A} \leq \vec{0} \land \vec{\pi}\transpose \vec{b} > 0 \\
\Leftrightarrow & \exists \vec{x} \in \mathbb{R}_+^n. \, \mat{A} \vec{x} \geq \vec{b} \qquad \land \forall \vec{\pi} \in \mathbb{R}_+^n. \, \neg \left( \vec{\pi}\transpose \mat{A} \leq \vec{0} \land \vec{\pi}\transpose \vec{b} > 0 \right)\\
\Leftrightarrow & \exists \vec{x} \in \mathbb{R}_+^n. \, \mat{A} \vec{x} \geq \vec{b} \qquad \land \forall \vec{\pi} \in \mathbb{R}_+^n. \, \vec{\pi}\transpose \mat{A} > \vec{0} \lor \vec{\pi}\transpose \vec{b} \leq 0 \\
\Rightarrow & \exists \vec{x} \in \mathbb{R}_+^n. \, \exists \vec{\pi} \in \mathbb{R}_+^n. \, \vec{\pi}\transpose \mat{A} \vec{x} \geq \vec{\pi}\transpose \vec{b}
\end{aligned}
\end{equation}

Furthermore, from the infeasibility of \RMP{} we can also derive the following statement:
\begin{equation}
\begin{aligned}
& \left( \forall \vec{\pi} \in \mathbb{R}_+^n. \, \vec{\pi}\transpose \mat{A} > \vec{0} \lor \vec{\pi}\transpose \vec{b} \leq 0 \right) \land \left( \exists \vec{\pi} \in \mathbb{R}_+^n. \, \vec{\pi}\transpose \mat{A}_{\vert \indexset{X}'} \leq \vec{0} \land \vec{\pi}\transpose \vec{b} > 0 \right) \\
\Rightarrow & \left( \neg \forall \vec{\pi} \in \mathbb{R}_+^n. \, \vec{\pi}\transpose \vec{b} \leq 0 \right) \land \left( \exists \vec{\pi} \in \mathbb{R}_+^n. \, \vec{\pi}\transpose \mat{A}_{\vert \indexset{X}'} > \vec{0} \right)
\end{aligned}
\end{equation}

Therefore, there is some variable $\vec{x} \in \indexset{X} \setminus \indexset{X}'$ such that its column $\vec{a}_\vec{x} \coloneqq \mat{A}_{\vert \{\vec{x}\}}$ is $\vec{\pi}\transpose \vec{a}_\vec{x} > 0$ for some $\vec{\pi} \in \mathbb{R}_+^n$.

This process of finding corresponding columns $\vec{a}_\vec{x}$ to add to the \RMP{} can be formalized as a pricing problem with cost coefficients $c_\vec{x} = 0$:
\begin{equation}
\operatorname{F}(\vec{\pi}) = \underset{x \in \indexset{X}}{\min} \, -\vec{\pi}\transpose \vec{a}_x
\end{equation}

We can add all solutions $\vec{x}$ with a solution value of $\operatorname{F}(\vec{\pi}) > 0$ to $\indexset{X}'' \coloneqq \indexset{X}' \cup \{\vec{x}_i\}$, thus turning any infeasible \RMP{} feasible.

\subsection{Reduced Cost Pricing}\label{sec:cg_bp_cg_reduced}

\subsection{Column Generation Algorithm}\label{sec:cg_bp_cg_alg}


\section{Dantzig-Wolfe Reformulation}\label{sec:cg_bp_dwr}
The column generation algorithm presented in section \ref{sec:cg_bp_cg} is especially practical when we can directly formulate our optimization problem using a master and a pricing problem. Oftentimes, however, we do not have these constructions readily available. Instead, many problems are given in the more general form of a \LP{}. Using the Dantzig-Wolfe reformulation, we can automatically transform such a \LP{} into a master and pricing problem, allowing us to apply column generation. In this section, we will introduce this technique and show how it can be used to solve a \LP{}.

\begin{equation}
\begin{aligned}
z^*_\LP{} = &\min & \vec{c}\transpose \vec{x} & & & \\
&\st & \mat{A} \vec{x} & \geq \vec{b} & \left[\vec{\sigma}_\vec{b}\right] & \\
&& \mat{D} \vec{x} & \geq \vec{d} & \left[\vec{\sigma}_\vec{d}\right] & \\
&& \vec{x} & \geq \vec{0}
\end{aligned}
\end{equation}

Take the above \LP{} as an example. The solution space if this \LP{}, defined by its constraints, can also be viewed as the intersection of the following two polyhedra:
\begin{equation}
\begin{aligned}
\polyhedron{A} &\coloneqq \left\{ \vec{x} \geq \vec{0} \mid \mat{A} \vec{x} \geq \vec{b} \right\} &\neq \emptyset \\
\polyhedron{D} &\coloneqq \left\{ \vec{x} \geq \vec{0} \mid \mat{D} \vec{x} \geq \vec{d} \right\} &\neq \emptyset
\end{aligned}
\end{equation}

After applying the Nemhauser-Wolsey Theorem (Theorem \ref{th:nemhauser-wolsey}) on polyhedron $\polyhedron{D}$, we can reformulate the \LP{} using $\polyhedron{D}$'s extreme points $\{\vec{x}_p\}_{p \in P}$ and extreme rays $\{\vec{x}_r\}_{r \in R}$. For this, we substitute the original variables $\vec{x}$ with these extreme points and extreme rays using:
\begin{equation}
\begin{aligned}
\vec{x} &= \sum_{p \in P} \vec{x}_p \lambda_p + \sum_{r \in R} \vec{x}_r \lambda_r \\
\vec{c}\transpose \vec{x} &= \sum_{p \in P} \vec{c}\transpose \vec{x}_p \lambda_p + \sum_{r \in R} \vec{c}\transpose \vec{x}_r \lambda_r \\
\mat{A} \vec{x} &= \sum_{p \in P} \mat{A} \vec{x}_p \lambda_p + \sum_{r \in R} \mat{A} \vec{x}_r \lambda_r
\end{aligned}
\end{equation}
Let us also use the following shorthand notations:
\begin{equation}
\begin{aligned}
c_p &\coloneqq \vec{c}\transpose \vec{x}_p
&c_r &\coloneqq \vec{c}\transpose \vec{x}_r \\
\vec{a}_p &\coloneqq \mat{A} \vec{x}_p
&\vec{a}_r &\coloneqq \mat{A} \vec{x}_r
\end{aligned}
\end{equation}

As a result, we have obtained a new \MP{} equivalent to the \LP{}:
\begin{equation}
\begin{aligned}
z^*_\MP{} = &\min & \sum_{p \in P} c_p \lambda_p &+ &\sum_{r \in R} c_r \lambda_r & & & \\
&\st & \sum_{p \in P} \vec{a}_p \lambda_p &+ &\sum_{r \in R} \vec{a}_r \lambda_r & \geq \vec{b} & \left[\vec{\pi}_\vec{b}\right] \\
&& \sum_{p \in P} \lambda_p & & & = 1 & \left[\pi_0 \right] & \\
&& \lambda_p & & & \geq 0 & & \forall p \in P \\
&& & & \lambda_r & \geq 0 & & \forall r \in R \\
&& \sum_{p \in P} \vec{x}_p \lambda_p &+ &\sum_{r \in R} \vec{x}_r \lambda_r & = \vec{x} \geq \vec{0} & &
\end{aligned}
\end{equation}
In this formulation, the last constraint corresponds to transforming a solution of the \MP{} using the $\lambda$ variables back into a solution of the original \LP{}. As this constraint is not otherwise involved in the optimization, it is often omitted during the solving stages and only used afterwards to reconstruct a solution using the original $\vec{x}$ variables.

As the number of extreme points and extreme rays of $\polyhedron{D}$ might be huge, it is most often than not practically infeasible to solve the \MP{} directly. Instead, using this setup, we can easily generate these columns on the fly using column generation. For this, we need a subproblem that finds (improving) columns for the \MP{}, i.e. extreme points and extreme rays of $\polyhedron{D}$. We can easily formulate this pricing problem as follows:
\begin{equation}
\begin{aligned}
z^*_\SP{} = &\min & \left( \vec{c}\transpose - \vec{\pi}_\vec{b}\transpose \mat{A} \right) \vec{x} - \pi_0 & & \\
&\st & \mat{D} \vec{x} & \geq \vec{d} & \left[\vec{\pi}_\vec{d}\right] \\
&& \vec{x} & \geq \vec{0}
\end{aligned}
\end{equation}

We start of by solving the \RMP{} using a subset of the extreme points $P' \subset P$ and extreme rays $R' \subset R$, giving us the dual values $\vec{\pi}_\vec{b}$ and $\pi_0$ for the \SP{}. Solving this \SP{} to optimality then leads a solution $\vec{x}^*$ with objective value $z^*_\SP{}$. The value of $z^*_\SP{}$ is now the deciding factor whether we add a column to \RMP{}, and if so, which column we add:
\begin{itemize}
\item If $-\infty < z^*_\SP{} < 0$, $\vec{x}^*$ is an extreme point $\vec{x}_p, p \in P \setminus P'$, and we add column $\begin{bmatrix} \vec{c}\transpose \vec{x}^* \\ \mat{A} \vec{x}^* \\ 1 \end{bmatrix}$ to the \RMP{}.
\item If $z^*_\SP{} = -\infty$, $\vec{x}^*$ is an extreme ray $\vec{x}_r, r \in R \setminus R'$, and we add column $\begin{bmatrix} \vec{c}\transpose \vec{x}^* \\ \mat{A} \vec{x}^* \\ 0 \end{bmatrix}$ to the \RMP{}.
\item If $z^*_\SP{} \geq 0$, there exists no improving column for the \RMP{}, thus the column generation algorithm terminates.
\end{itemize}

While in theory it does not matter how we group the constraints of our original formulation \LP{} for the Dantzig-Wolfe reformulation, since all groupings result in equivalent optimal solutions, in practice the choice of grouping can have a significant impact on the performance of the column generation algorithm. Since most of the time many iterations of the column generation algorithm are required to find an optimal solution, ideally one wants the \SP{} to be efficiently solvable. Many highly efficient algorithms for specific optimization problems exist, and by grouping constraints in a way that the \SP{} corresponds to such structures, one can leverage these algorithms to solve the \SP{} efficiently. Thankfully, there are ways of finding such groupings automatically, although this goes beyond the scope of this thesis.

\section{Dantzig-Wolfe Reformulation for Mixed Integer Programs}\label{ch:cg_bp_ip}
\section{Several and Identical Subproblems}\label{sec:cg_bp_idsp}
Many applications are composed of different families of variables and constraints, which can be decomposed into several distinct subproblems. Column generation can be adapted to this scenario, where we have a set $K$ of subproblems $\SP{k}$ generating variables $\vec{x}^k \in \indexset{X}^k$ \cite{thebook}. Our \MP{} is then defined as:

\begin{equation}
\begin{aligned}
z^*_\MP{} = &\min & \sum_{k \in K}\sum_{\vec{x}^k \in \indexset{X}^k} c_{\vec{x}^k} \lambda_{\vec{x}^k} & & & \\
&\st & \sum_{k \in K}\sum_{\vec{x}^k \in \indexset{X}^k} \vec{a}_{\vec{x}^k} \lambda_{\vec{x}^k} & \geq \vec{b} & \left[\vec{\pi}\right] & \\
&& \lambda_\vec{x} & \geq 0 & & \forall k \in K. \forall {\vec{x}^k} \in \indexset{X}^k
\end{aligned}
\end{equation}

All subproblems $\SP{k}$ now use the same dual values $\vec{\pi}$, and the pricing problem for each subproblem $\SP{k}$ is defined as:

\begin{equation}
\begin{aligned}
z^*_\SP{k} = \underset{\vec{x}^k \in \indexset{X}^k}{\min} \, c_{\vec{x}^k} - \vec{\pi}\transpose \vec{a}_{\vec{x}^k}
\end{aligned}
\end{equation}

The column generation algorithm from Section \ref{sec:cg_bp_cg_alg} proceeds as before, with the adaptation that it terminates only when \textit{all} subproblems $\SP{k}$ produce columns with non-negative reduced costs.

This idea of having several subproblems generating columns for the master problem can also be applied to Dantzig-Wolfe reformulated \LP{}s and \IP{}s. Recall that we find two groups of constraints:

\begin{equation}
\begin{aligned}
\polyhedron{A} &\coloneqq \left\{ \vec{x} \geq \vec{0} \mid \mat{A} \vec{x} \geq \vec{b} \right\} &\neq \emptyset \\
\polyhedron{D} &\coloneqq \left\{ \vec{x} \geq \vec{0} \mid \mat{D} \vec{x} \geq \vec{d} \right\} &\neq \emptyset
\end{aligned}
\end{equation}

In many applications, the coefficient matrix $\mat{D}$ has a block diagonal structure \cite{thebook}:

\begin{equation}
\mat{D} = \begin{bmatrix} \mat{D}^1 & & \\ & \ddots & \\ & & \mat{D}^{\abs{K}} \end{bmatrix}
\qquad\text{and}\qquad
\vec{d} = \begin{bmatrix} \vec{d}^1 \\ \vdots \\ \vec{d}^{\abs{K}} \end{bmatrix}
\end{equation}

Each of these $k \in K$ blocks can be considered its own subproblem independent of others. Therefore, another way of writing the \MP{} for Dantzig-Wolfe reformulated \LP{}s is (analogous for convexification and discretization of \IP{}s):

\begin{equation}
\begin{aligned}
z^*_\MP{} = &\min & \sum_{k \in K}\sum_{p \in P^k} c_p^k \lambda_p^k & + & \sum_{k \in K}\sum_{r \in R^k} c_r^k \lambda_r^k & & & \\
&\st & \sum_{k \in K}\sum_{p \in P^k} \vec{a}_p^k \lambda_p^k & + & \sum_{k \in K}\sum_{r \in R^k} \vec{a}_r^k \lambda_r^k & \geq \vec{b} & \left[\vec{\pi}_\vec{b}\right] & \\
&& \sum_{p \in P^k} \lambda_p^k & & & = 1 & \left[\pi_0^k \right] & \forall k \in K \\
&& \lambda_p^k & & & \geq 0 & & \forall k \in K, \forall p \in P^k \\
&& & & \lambda_r^k & \geq 0 & & \forall k \in K, \forall r \in R^k \\
&& \sum_{p \in P^k} \vec{x}_p^k \lambda_p^k & + & \sum_{r \in R^k} \vec{x}_r^k \lambda_r^k & = \vec{x}^k \geq \vec{0} & & \forall k \in K
\end{aligned}
\end{equation}

Each subproblem $\SP{k}$ is given by:

\begin{equation}
\begin{aligned}
z^*_\SP{k} = &\min & \left( \vec{c}^{k \intercal} - \vec{\pi}_\vec{b}\transpose \mat{A}^k \right) \vec{x}^k - \pi_0^k & & \\
&\st & \mat{D}^k \vec{x}^k & \geq \vec{d}^k & \left[\vec{\pi}_\vec{d}^k\right] \\
&& \vec{x}^k & \geq \vec{0}
\end{aligned}
\end{equation}

Now, consider the case where all blocks are equal, i.e., $\mat{D}^1 = \ldots = \mat{D}^{\abs{K}} = \mat{D}$ and $\vec{d}^1 = \ldots = \vec{d}^{\abs{K}} = \vec{d}$. In this case, all subproblems $\SP{k}$ are identical, generating new columns from the same set of extreme points and extreme rays. This implies that in \MP{}, different $\lambda_p^k$ ($\lambda_r^k$) variables for different $k$ correspond to the same extreme point $\vec{x}_p$ ($\vec{x}_r$), which is redundant and could slow down the solving process \cite{thebook}. In a process called \textbf{aggregation}, we can improve upon this by eliminating this redundancy:

\begin{equation}
\lambda_p \coloneqq \sum_{k \in K} \lambda_p^k, \; \forall p \in P
\qquad\text{and}\qquad
\lambda_r \coloneqq \sum_{k \in K} \lambda_r^k, \; \forall r \in R
\end{equation}

Substituting these aggregated variables in \MP{} yields:

\begin{subequations}
\begin{alignat}{11}
z^*_\MP{} = &\min & \sum_{p \in P} c_p \lambda_p & + & \sum_{r \in R^k} c_r \lambda_r & & & \\
&\st & \sum_{p \in P} \vec{a}_p \lambda_p & + & \sum_{r \in R^k} \vec{a}_r \lambda_r & \geq \vec{b} & \left[\vec{\pi}_\vec{b}\right] & \\
&& \sum_{p \in P} \lambda_p & & & = \abs{K} & \left[\pi_{agg} \right] & \\
&& \lambda_p & & & \geq 0 & & \forall p \in P \\
&& & & \lambda_r & \geq 0 & & \forall r \in R \\
&& \sum_{k \in K} \lambda_p^k & & & = \lambda_p & & \forall p \in P \label{eq:cg_bp_idsp_disagg1}\\
&& & & \sum_{k \in K} \lambda_r^k & = \lambda_r & & \forall r \in R \\
&& \sum_{p \in P} \lambda_p^k & & & = 1 & & \forall k \in K \\
&& \lambda_p^k & & & \geq 0 & & \forall k \in K, \forall p \in P \\
&& & & \lambda_r^k & \geq 0 & & \forall k \in K, \forall r \in R \label{eq:cg_bp_idsp_disagg2}\\
&& \sum_{p \in P} \vec{x}_p \lambda_p^k & + & \sum_{r \in R} \vec{x}_r \lambda_r^k & = \vec{x}^k \geq \vec{0} & & \forall k \in K \label{eq:cg_bp_idsp_agg}
\end{alignat}
\end{subequations}

Columns for this \MP{} are generated by the following subproblem:

\begin{equation}
\begin{aligned}
z^*_\SP{agg} = &\min & \left( \vec{c}\transpose - \vec{\pi}_\vec{b}\transpose \mat{A} \right) \vec{x} - \pi_{agg} & & \\
&\st & \mat{D} \vec{x} & \geq \vec{d} & \left[\vec{\pi}_\vec{d}\right] \\
&& \vec{x} & \geq \vec{0}
\end{aligned}
\end{equation}

The constraints \eqref{eq:cg_bp_idsp_disagg1} to \eqref{eq:cg_bp_idsp_disagg2} disaggregate a solution for the aggregated variables back into the master variables for each subproblem, which are used to compute a solution to the original formulation using the original variables $\vec{x}^k$. For this reason, the constraints from \eqref{eq:cg_bp_idsp_disagg1} onwards may be omitted during the column generation algorithm. This statement also holds for Dantzig-Wolfe reformulated \IP{}s using the convexification approach, where the only difference in \MP{} are the integrality conditions on $\vec{x}^k$ in constraint \eqref{eq:cg_bp_idsp_agg}. In convexification, however, we can only ensure the integrality of the original solution by branching on the integer original variables with fractional value. Therefore, we constantly need to reintroduce the disaggregated master variables to project a solution of \RMP{} to an original solution.

Discretization, however, offers a powerful alternative. Its \MP{} for identical subproblems looks as follows:

\begin{subequations}
\begin{alignat}{11}
z^*_\MP{} = &\min & \sum_{p \in P} c_p \lambda_p & + & \sum_{r \in R^k} c_r \lambda_r & & & \\
&\st & \sum_{p \in P} \vec{a}_p \lambda_p & + & \sum_{r \in R^k} \vec{a}_r \lambda_r & \geq \vec{b} & \left[\vec{\pi}_\vec{b}\right] & \\
&& \sum_{p \in P} \lambda_p & & & = \abs{K} & \left[\pi_{agg} \right] & \\
&& \lambda_p & & & \in \mathbb{Z}_+ & & \forall p \in P \\
&& & & \lambda_r & \in \mathbb{Z}_+ & & \forall r \in R \\
&& \sum_{k \in K} \lambda_p^k & & & = \lambda_p & & \forall p \in P \label{eq:cg_bp_idsp_disagg3}\\
&& & & \sum_{k \in K} \lambda_r^k & = \lambda_r & & \forall r \in R \\
&& \sum_{p \in P} \lambda_p^k & & & = 1 & & \forall k \in K \\
&& \lambda_p^k & & & \in \mathbb{Z}_+ & & \forall k \in K, \forall p \in P \\
&& & & \lambda_r^k & \in \mathbb{Z}_+ & & \forall k \in K, \forall r \in R \label{eq:cg_bp_idsp_disagg4}\\
&& \sum_{p \in P} \vec{x}_p \lambda_p^k & + & \sum_{r \in R} \vec{x}_r \lambda_r^k & = \vec{x}^k \in \mathbb{Z}_+^n & & \forall k \in K \label{eq:cg_bp_idsp_agg2}
\end{alignat}
\end{subequations}

In Section \ref{sec:cg_bp_ip_discretization}, we have observed that the integrality constraints on the original variables $\vec{x}^k$ are already enforced by ensuring the integrality of the disaggregated master variables $\lambda_p^k$ and $\lambda_r^k$. In the case of identical subproblems, we can go a step further and also neglect the integrality constraints on the disaggregated master variables, as those are implied by the integrality of the aggregated variables $\lambda_p$ and $\lambda_r$ \cite{thebook}. Therefore, during the entire solving process, we can omit the constraints \eqref{eq:cg_bp_idsp_disagg3} to \eqref{eq:cg_bp_idsp_agg2} entirely.

On a final note, it is possible to have both identical and differing subproblems in the same \MP{}. In this case, we introduce classes $C$ of identical subproblems, use one column generator per class, and aggregate the variables within each class.

\section{Branch-and-Price}\label{sec:cg_bp_bp}
In Section \ref{sec:cg_bp_ip} we have seen how to reformulate an integer program into a master and pricing problem, specifically focusing on the integrality conditions. In this section, we will dive into how we then solve such an integer master program using column generation. First, let us remember what branching is for. Recall, that often we cannot solve an integer problem directly. Instead, we rely on the \LP{} relaxations of the problem which in turn can be solved by algorithms such as the simplex method. An optimal solution of the \LP{} relaxation might have some fractional values for the integer variables, i.e. produce infeasible solutions for the \IP{}. To overcome this, we branch on these fractional variables, creating subproblems, which explicitly cut off these fractional solutions. By recursively solving these subproblems, we eventually find an optimal integer solution. This process is widely known as \textbf{branch-and-bound}.

In the context of column generation for integer master programs, we proceed similarly: first, we relax the integrality constraints of the master problem, which allows us to solve the relaxation using column generation to optimality. Then, we check if the integrality conditions are satisfied. If not, we must cut off the fractional solution by branching. Combining branching with column generation, we obtain the term \textbf{branch-and-price}.

We have gotten to know two distinct approaches of reformulating an \IP{} into a (integer) master and pricing problem: convexification (Section \ref{sec:cg_bp_ip_convexification}) and discretization (Section \ref{sec:cg_bp_ip_discretization}). Since we require integrality of the original variables in both approaches, it is always possible to branch on fractional solutions of the original variables. We have seen, however, that discretization additionally introduces integrality constraints on the master variables which in turn imply integrality of the original variables. Therefore, in discretization, we can branch on the master variables as well. In the following, we will discuss both approaches in more detail.

\subsection{Branching on the Original Variables}\label{sec:cg_bp_bp_branching_original}
Assume we have a fractional solution $\vec{x}_\MP{}^*$ to the relaxed restricted master problem \RMP{}, i.e. there is some $x_j^* \not\in \mathbb{Z}$ for some integer variable $x_j$. Then we can cut off this fractional solution by creating two subbranches (\textbf{dichotomous branching}), one where $x_j \leq \floor{x_j^*}$ and one where $x_j \geq \ceil{x_j^*}$. In the branch-and-price context, there are actually two ways to enforce this branching decision:

\subsubsection{Branching in the Master Problem}
Recall that the \MP{} includes the following constraint:
\begin{equation}
\sum_{p \in P} \vec{x}_p \lambda_p + \sum_{r \in R} \vec{x}_r \lambda_r = \vec{x} \in \mathbb{Z}_+^n
\end{equation}
Obviously, this constraint is now violated in the case of variable $x_j$. We can enforce the branching decision $x_j \leq \floor{x_j^*}$ by adding the following constraint to the \MP{} (analogous for the up-branch):
\begin{equation}
\sum_{p \in P} x_{pj} \lambda_p + \sum_{r \in R} x_{rj} \lambda_r \leq \floor{x_j^*} \quad \left[{\color{blue} \alpha_j }\right]
\end{equation}
In order to keep generating only improving columns after branching, we must consider the dual variable $\alpha_j$ in the pricing problem:
\begin{equation}
\begin{aligned}
z^*_\SP{} = &\min & \left( \vec{c}\transpose - \vec{\pi}_\vec{b}\transpose \mat{A} \right) \vec{x} {\color{blue} - \alpha_j x_j} - \pi_0 & \\
&\st & \mat{D} \vec{x} & \geq \vec{d} \\
&& \vec{x} & \in \mathbb{Z}_+^n
\end{aligned}
\end{equation}

\subsubsection{Branching in the Pricing Problem}
Alternatively, we may add the branching decision directly to the pricing problem:
\begin{equation}
\begin{aligned}
z^*_\SP{} = &\min & \left( \vec{c}\transpose - \vec{\pi}_\vec{b}\transpose \mat{A} \right) \vec{x} - \pi_0 & \\
&\st & \mat{D} \vec{x} & \geq \vec{d} \\
&& {\color{blue} x_j} & {\color{blue} \leq \floor{x_j^*}}\\
&& \vec{x} & \in \mathbb{Z}_+^n
\end{aligned}
\end{equation}
Unfortunately, the \RMP{} might already contain generated columns that violate the branching decision. To ensure correctness of this implementation of the branching decision, we must forbid all existing columns with $x_j > \floor{x_j^*}$ from being part of the solution in the master. This could be achieved by removing such columns altogether, or by adding the following constraint to the \MP{}:
\begin{equation}
\sum_{p \in P: x_{pj} > \floor{x_j^*}} \lambda_p + \sum_{r \in R: x_{rj} > \floor{x_j^*}} \lambda_r = 0
\end{equation}

\subsection{Branching on the Master Variables}\label{sec:cg_bp_bp_branching_master}
\subsubsection{Vanderbeck's Geneic Branching Scheme}
\subsubsection{Special Case: Ryan-Foster Branching}

\section{Branch-Price-and-Cut}\label{sec:cg_bp_bpc}
\section{Dual Value Stabilization}\label{sec:cg_bp_dvs}
During column generation, it has been observed that the dual values oscillate erratically, which means it takes more column generation iterations to generate columns that are considered profitable. One can, however, stabilize the dual values, decreasing such oscillations, resulting in a significant performance improvement of the column generation algorithm. This section covers the fundamentals of the hybridization of the dynamic alpha-schedule stabilization with an ascent method, as proposed by Possea et al., for which we will introduce the building blocks step by step. For more detail, we refer to \cite{pessoa2013out,pessoa2018automation}.

// TODO

\cleardoublepage
\chapter{\SCIP{} Optimization Suite}\label{ch:tools}
The \SCIP{} Optimization Suite is a comprehensive collection of software tools designed to address a wide range of mathematical optimization problems. Central to this suite is the \SCIP{} (Solving Constraint Integer Programs) framework \cite{achterberg2007constraint, achterberg2009scip}, which serves both as a branch-price-and-cut solver and a development platform manly for mixed-integer programming (\MIP{}) and constraint integer programming (\CIP{}).

In addition to \SCIP{} itself, the \SCIP{} Optimization Suite includes several other tools that complement its functionality, such as aa \LP{} solver, a modeling language, and a parallelization layer for exploiting multi-core and distributed computing resources. For further information, we refer the reader to the official \SCIP{} website\footnote{\url{https://www.scipopt.org/}} as well as \cite{bolusani2024scip}.

\section{GCG}\label{sec:tools_gcg}
The Generic Column Generation (\GCG{}) solver developed by Gamrath et al. \cite{gamrath2010generic} is a solver implemented using the \SCIP{} framework, specifically designed to implement the Dantzig-Wolfe reformulation and solve optimization problems using column generation and branch-price-and-cut techniques. It works by detecting a suitable decomposition of the problem, reformulating it as a master problem with a set of subproblems using a Dantzig-Wolfe reformulation (see Section \ref{sec:cg_bp_dwr}), and then solving the master problem using column generation (see Section \ref{sec:cg_bp_cg}). To solve \MIP{}s, \GCG{} utilizes the branch-price-and-cut algorithm (Section \ref{sec:cg_bp_bp}), providing multiple branching strategies, including Vanderbeck's generic branching scheme (see Section \ref{sec:cg_bp_bp_branching_generic}). Additionally, \GCG{} supports the stabilization of dual values to improve the convergence of the column generation process (see Section \ref{sec:gm_dvs}).

\cleardoublepage
\chapter{Component Bound Branching}\label{ch:cmpbnd}
In this chapter, we introduce the \textbf{component bound branching rule} (\texttt{COMPBND}) for branching on the master variables of the discretized reformulation of any type of bounded \IP{}. This new branching rule builds upon the fundamental ideas of Vanderbeck's generic branching scheme (Section \ref{sec:cg_bp_bp_branching_generic}), aiming to provide a simpler alternative to branching on component bounds. We will begin by demonstrating how to enforce component bounds to create a binary branch-and-bound search tree. Next, we will delve into the algorithm responsible for determining suitable branching decisions. Finally, we will compare and contrast Vanderbeck's generic branching scheme with our new approach, highlighting their similarities and differences.

\section{Overview of the branching scheme}\label{sec:cmpbnd_overview}
As discussed in Section \ref{sec:cg_bp_bp_branching_master}, given a fractional master solution $\vec{\lambda}^*_\RMP{}$, we can always find a subset $\emptyset \subset Q' \subset Q \coloneqq \ddot{P}$ such that:

\begin{equation}\label{eq:compbnd_branching_master}
\sum_{q \in Q'} \lambda_q^* \eqqcolon K \not\in \mathbb{Z}
\end{equation}

This allows us to eventually enforce the integrality of $\vec{\lambda}_\MP{}$, for example, by adding one of the following branching constraints to each child node:

\begin{equation}
\begin{aligned}
\sum_{q \in Q'} \lambda_q \leq \lfloor K \rfloor \quad \left[\gamma\right] \\
\sum_{q \in Q'} \lambda_q \geq \lceil K \rceil \quad \left[\gamma\right]
\end{aligned}
\end{equation}

Adding such constraints to the master problem requires us to modify the pricing problem in the following way:

\begin{equation}
\begin{aligned}
z^*_\SP{} = &\min & \left( \vec{c}\transpose - \vec{\pi}_\vec{b}\transpose \mat{A} \right) \vec{x} {\color{blue} - \gamma y} - \pi_0 & \\
&\st & \mat{D} \vec{x} & \geq \vec{d} \\
&& {\color{blue} y = 1} & {\color{blue} \Leftrightarrow \vec{x} \in Q'} \\
&& \vec{x} & \in \mathbb{Z}_+^n \\
&& {\color{blue} y} & {\color{blue} \in \{0, 1\}}
\end{aligned}
\end{equation}
where $y$ becomes the column entry for the row added to the master, and $y = 1 \Leftrightarrow \vec{x} \in Q'$ is expressible using a finite set of linear constraints.

To find such a $Q'$ that is expressible in \SP{}, Vanderbeck proposes to use bounds on the components of the columns. Similarly, in our branching scheme, we also find such component bounds. Let us reiterate the notation introduced for Vanderbeck's branching scheme in Section \ref{sec:cg_bp_bp_branching_generic}:

\begin{equation}
B \coloneqq \left( x_i, \eta, v \right) \in \{x_i \mid 1 \leq i \leq n\} \times \{\leq, \geq\} \times \mathbb{Z}
\end{equation}
\begin{equation}
\bar{B} \coloneqq \left( x_i, \bar{\eta}, v \right), \bar{\eta} \coloneqq \begin{cases} \leq & \text{if } \eta = \geq \\ \geq & \text{if } \eta = \leq \end{cases}
\end{equation}

We define a component bound sequence as follows:

\begin{equation}
S \coloneqq \left\{ \left( x_{i,1}, \eta_1, v_1 \right), \dots, \left( x_{j,k}, \eta_k, v_k \right) \right\} \in 2^{\{x_i \mid 1 \leq i \leq n\} \times \{\leq, \geq\} \times \mathbb{Z}}
\end{equation}
and restrictions of $S$ to only upper bounds $\bar{S}$ and lower bounds $\ubar{S}$ respectively:

\begin{equation}
\begin{aligned}
\bar{S} &\coloneqq \left\{ \left( x_{i}, \leq, v \right) \mid \left( x_{i}, \leq, v \right) \in S \right\} \\
\ubar{S} &\coloneqq \left\{ \left( x_{i}, \geq, v \right) \mid \left( x_{i}, \geq, v \right) \in S \right\}
\end{aligned}
\end{equation}

We continue using the following shorthand notation:

\begin{equation}
\eta(a, v) \Leftrightarrow
\begin{cases}
a \leq v & \text{if } \eta = \leq \\
a \geq v & \text{if } \eta = \geq
\end{cases}
\end{equation}

Similar to Vanderbeck's branching, we can find such a subset $Q'$ by finding a component bound sequence $S$ such that:

\begin{equation}
\sum_{q \in Q(S)} \lambda_q^* \eqqcolon K \not\in \mathbb{Z}
\end{equation}
where $Q(S) \coloneqq \{q \in Q \mid \forall (x_i, \eta, v) \in S. \eta(x_{qi}, v)\}$.

Proof \ref{pr:cg_bp_bp} shows that such an $S$ always exists if the master solution is not integral. After obtaining such an $S$, we create two child nodes, the down- and up-branches, by first adding the branching decision to the master problem:

\begin{multicols}{2}
\noindent
\begin{minipage}{\linewidth}
\setlength{\belowdisplayskip}{0pt} \setlength{\belowdisplayshortskip}{0pt}
\setlength{\abovedisplayskip}{0pt} \setlength{\abovedisplayshortskip}{0pt}
\begin{equation*}
\sum_{q \in Q(S)} \lambda_q \leq \lfloor K \rfloor \quad \left[\gamma_{\downarrow} \leq 0\right]
\end{equation*}
\end{minipage}

\columnbreak

\noindent
\begin{minipage}{\linewidth}
\setlength{\belowdisplayskip}{0pt} \setlength{\belowdisplayshortskip}{0pt}
\setlength{\abovedisplayskip}{0pt} \setlength{\abovedisplayshortskip}{0pt}
\begin{equation}
\sum_{q \in Q(S)} \lambda_q \geq \lceil K \rceil \quad \left[\gamma_{\uparrow} \geq 0\right]
\end{equation}
\end{minipage}
\end{multicols}

We now must ensure that newly priced columns $x_{q'}$ are assigned a coefficient of $y = 1$ for the branching decision if $q' \in Q(S)$, i.e., if $\forall (x_i, \eta, v) \in S. \eta(x_{q'i}, v)$ and otherwise $y = 0$. We achieve this by introducing additional binary variables $\bar{y}_s, \ubar{y}_{s'}$ for each $B_s \in \bar{S}$ and for each $B_{s'} \in \ubar{S}$ respectively, along with the following constraints, in the \SP{} \cite{thebook}:

\begin{equation}
\begin{aligned}
y = 1 &\Leftrightarrow \sum_{B_s \in \bar{S}} \bar{y}_s + \sum_{B_s \in \ubar{S}} \ubar{y}_s = \abs{S} &\\
\bar{y}_s = 1 &\Leftrightarrow x_s \leq v_s & \forall B_s \in \bar{S} \\
\ubar{y}_s = 1 &\Leftrightarrow x_s \geq v_s & \forall B_s \in \ubar{S} \\
y &\in \{0, 1\} & \\
\bar{y}_s &\in \{0, 1\} & \forall B_s \in \bar{S} \\
\ubar{y}_s &\in \{0, 1\} & \forall B_s \in \ubar{S}
\end{aligned}
\end{equation}

What remains is to express all logical equivalences using a finite set of linear constraints. For this, the following observations are crucial \cite{thebook}:

\begin{itemize}
\item In the down branch, since $-\gamma_{\downarrow} \geq 0$, $y$ naturally takes the value $0$ and so do all $\bar{y}_s$ and $\ubar{y}_{s'}$. Thus, in the down branch, we need to force all $\bar{y}_s$ and $\ubar{y}_{s'}$ to $1$ if the corresponding component bounds are satisfied, and force $y$ to $1$ if all $\bar{y}_s$ and $\ubar{y}_{s'}$ equal $1$.
\item In the up branch, the opposite is the case: since $-\gamma_{\uparrow} \leq 0$, $y$ and all $\bar{y}_s, \ubar{y}_{s'}$ naturally take the value $1$, requiring us to force all $\bar{y}_s$ and $\ubar{y}_{s'}$ to $0$ if their corresponding component bounds are not satisfied, and force $y$ to $0$ if any of the $\bar{y}_s, \ubar{y}_{s'}$ equals $0$.
\end{itemize}

Given that we require a bounded \IP{} to begin with, let us denote the lower and upper bounds of a variable $x_i$ as $\text{lb}_i$ and $\text{ub}_i$ respectively. Using the above observations, we can now express the logical equivalences mandated by the branching decision as follows \cite{thebook}:

\begin{multicols}{2}
\noindent
\begin{minipage}{0.95\linewidth}
\setlength{\belowdisplayskip}{0pt} \setlength{\belowdisplayshortskip}{0pt}
\setlength{\abovedisplayskip}{4pt} \setlength{\abovedisplayshortskip}{4pt}
\begin{flalign*}
y &\geq 1 + \sum_{B_s \in \bar{S}} \bar{y}_s  + \sum_{B_s \in \ubar{S}} \ubar{y}_s - \abs{S} &
\end{flalign*}
\begin{flalign*}
\bar{y}_s &\geq \frac{(v_s + 1) - x_i}{(v_s + 1) - \text{lb}_i} &\forall B_s \in \bar{S} \\
\ubar{y}_s &\geq \frac{x_i - (v_s - 1)}{\text{ub}_i - (v_s - 1)} &\forall B_s \in \ubar{S}
\end{flalign*}
\end{minipage}

\columnbreak

\noindent
\begin{minipage}{\linewidth}
\setlength{\belowdisplayskip}{0pt} \setlength{\belowdisplayshortskip}{0pt}
\setlength{\abovedisplayskip}{0pt} \setlength{\abovedisplayshortskip}{0pt}
\begin{equation}
\begin{aligned}
y &\leq \bar{y}_s &\qquad \forall B_s \in \bar{S} \\
y &\leq \ubar{y}_s &\qquad \forall B_s \in \ubar{S} \\
\bar{y}_s &\leq \frac{\text{ub}_i - x_i}{\text{ub}_i - v_s} &\qquad \forall B_s \in \bar{S} \\
\ubar{y}_s &\leq \frac{x_i - \text{lb}_i}{v_s - \text{lb}_i} &\qquad \forall B_s \in \ubar{S}
\end{aligned}
\end{equation}
\end{minipage}
\end{multicols}

We have now successfully defined the branching decision in the master problem and the corresponding constraints in the pricing problem. Until we find an optimal integral solution of master variables, we will continue to branch using a suitable component bound sequence $S$, creating a binary search tree. In the next section, we present an algorithm responsible for finding such an $S$ given a fractional master solution $\vec{\lambda}^*_\RMP{}$.

\section{Separation Procedure}\label{sec:cmpbnd_separation}
\begin{definition}
The \textbf{fractionality of} $\vec{\lambda}^*_\RMP{}$ \textbf{ with respect to } $S$ is given by:
\begin{equation}\label{eq:cmpbnd_fractionality}
F_S = \sum_{q \in \mathcal{Q}(S)} \left( \lambda_q^* - \floor{\lambda_q^*} \right) \geq 0
\end{equation}
\end{definition}

When $S = \emptyset$, we have $Q(S) = Q$, and thus $F_S > 0$ since at least one $\lambda_q^*$ is fractional. In this case, $F_S \in \mathbb{Z}_+ \setminus \{0\}$ due to the convexity constraint $\sum_{q \in Q} \lambda_q = 1$ in the \MP{} (analogous in aggregated subproblems, see Section \ref{sec:cg_bp_idsp}).

In general, for any $S$ one of three cases can occur:
\begin{itemize}
\item	$F_S = 0$: $Q(S)$ contains no column with fractional $\lambda_q^*$. Thus, branching on $S$ would not cut off the current fractional solution $\vec{\lambda}^*_\RMP{}$. Adding further component bounds to $S$ would not change this.
\item	$a < F_S < a + 1, a \in \mathbb{Z}_+$. Using Equation \eqref{eq:cmpbnd_fractionality}, we can rewrite this as:
		\begin{equation}
		\sum_{q \in \mathcal{Q}(S)} \floor{\lambda_q^*} < \sum_{q \in \mathcal{Q}(S)} \lambda_q^* < \sum_{q \in \mathcal{Q}(S)} \floor{\lambda_q^*} + 1
		\end{equation}
		The sum $\sum_{q \in \mathcal{Q}(S)} \lambda_q^* \eqcolon K$ is fractional, enabling us to branch on $S$ (see Equation \eqref{eq:compbnd_branching_master}).
\item	$F_S \in \mathbb{Z}_+ \setminus \{0\}$. In this case, $\sum_{q \in \mathcal{Q}(S)} \lambda_q^* \in \mathbb{Z}_+$, and therefore branching on $S$ would not cut off the current fractional solution. However, using \ref{note:distinct_columns}, we can find two distinct columns $q_1, q_2 \in Q(S)$, i.e., where $x_{i,q_1} < x_{i,q_2}$ for some $i \in \{1, \dots, n\}$, such that $\lambda_{q_1}^*$ and $\lambda_{q_2}^*$ are fractional. Denote the rounded median of these two column entries as $v \coloneqq \floor{\frac{x_{i,q_1} + x_{i,q_2}}{2}}$. Since $x_{i,q_1} \leq v < v + 1 \leq x_{i,q_2}$, we can separate $q_1$ from $q_2$ by imposing a bound on the component $x_i$, i.e., expand $S$ to either $S_1$ or $S_2$, where:
		\begin{equation}
		\begin{aligned}
		S_1 &\coloneqq S \cup \{\left( x_i, \leq, v \right)\}\\
		S_2 &\coloneqq S \cup \{\left( x_i, \geq, v + 1 \right)\}
		\end{aligned}
		\end{equation}
		Note that $F_S = F_{S_1} + F_{S_2}$, thus we can always at least halve the fractionality of the current solution. Furthermore, both $Q(S_1)$ and $Q(S_2)$ are guaranteed to contain at least one fractional column, ensuring $F_{S_1}, F_{S_2} > 0$.
\end{itemize}

These observations suggest the following separation procedure: initialize $S^0 = \emptyset$, i.e., $F_{S^0} > 0$. While $F_{S^k} \in \mathbb{Z}_+ \setminus \{0\}$, find a component bound $x_i$ to branch on, yielding $S_1$ and $S_2$. Proceed with either as $S^{k+1}$. Finally, $F_{S^k}$ will be fractional, and we can branch on $S^k$ \cite{thebook}.

\begin{proposition}
At no iteration $k \geq 0$ will the separation procedure produce a component bound sequence $S^k$ with $F_{S^k} = 0$.
\end{proposition}

\begin{proof}
As previously discussed, $F_\emptyset > 0$, i.e., $S^0$ satisfies the proposition.

Assume $S^k$ satisfies the proposition, i.e., $F_{S^k} > 0$. If $F_{S^k} \not\in \mathbb{Z}_+$, the procedure terminates, and the proposition holds. Else $F_{S^k} \not\in \mathbb{Z}_+ \setminus \{0\}$. In this case, let us assume $F_{S^{k+1}} = 0$. Then $Q(S^{k+1})$ contains no fractional columns, which contradicts the design of $S^{k+1}$. By contradiction, $F_{S^{k+1}} > 0$ must hold, and by induction, the proposition holds.
\end{proof}

\begin{proposition}
Given that $\vec{\lambda}^*_\RMP{}$ contains finitely many non-zero values, the separation procedure will terminate after a finite number of iterations.
\end{proposition}

\begin{proof}
Let us denote the restriction of $Q(S)$ to the columns $q$ with fractional $\lambda_q^*$ as $Q_f(S)$. By our assumption $\abs{Q_f(S)} < \infty$. At each iteration $k$, we only remove columns from $Q_f(S^k)$, i.e., $\abs{Q_f(S^{k+1})} < \abs{Q_f(S^k)}$. Since $\abs{Q_f(S^0)} < \infty$, the separation procedure must terminate after a finite number of iterations.
\end{proof}

\subsection{Choice of Component Bounds}\label{sec:cmpbnd_separation_choice}
The separation procedure described above is not complete, as we have not yet defined which bounds we impose on which components. This choice can significantly impact the performance of the subsequent solving of the child nodes. In the worst case, the separation procedure will yield a component bound sequence $S$ for which $Q(S)$ only contains one column, i.e., dichotomous branching. Maintaining balance within the tree is generally beneficial, but the time required to find an optimal $S$ can grow arbitrarily large and must be traded off against improved performance that comes with a balanced tree. We propose the following two-staged approach:

In the first stage, using one or multiple heuristics, we recursively determine a set of valid component bound sequences $S_1, \dots, S_m$ for the current fractional master solution $\vec{\lambda}^*_\RMP{}$. For this, we adapt the previously described separation procedure to explore both options $S_1$ and $S_2$ whenever $F_S$ is integral. A first-stage heuristic is now only responsible for finding a separating component $x_i$ and bound value $v \in \mathbb{Z}$. Both the lower bound $x_i \leq v$ and the upper bound $x_i \geq v + 1$ will be explored further; we do not have to choose between them at this stage. In particular, we propose two heuristics for this first stage:

\begin{itemize}
\item	\texttt{MaxRangeMidrange} Heuristic: At each iteration $k$, we choose the component $x_i$ for which the components $x_{i,q}$ of the columns $q \in Q_f(S^k)$ are most spread out. We then bound $x_i$ by the midrange of these components. Formally, we define:
		\begin{equation*}
		\begin{aligned}
		max_j &\coloneqq \underset{q \in Q_f(S^k)}{\arg\max} \; x_{j,q} & \forall j \in \{1, \dots, n\}\\
		min_j &\coloneqq \underset{q \in Q_f(S^k)}{\arg\min} \; x_{j,q} & \forall j \in \{1, \dots, n\}\\
		x_i &= \underset{j \in \{1, \dots, n\}}{\arg\max} \; max_j - min_j & \\
		v &\coloneqq \frac{max_i - min_i}{2} &
		\end{aligned}
		\end{equation*}
\item	\texttt{MostDistinctMedian} Heuristic: At each iteration $k$, we choose the component $x_i$ for which the components $x_{i,q}$ of the columns $q \in Q_f(S^k)$ have the most distinct values. We then choose $v$ to be the median of these components.
\end{itemize}

In the second stage, another heuristic now scores every component bound sequence, and we continue branching using the highest scoring $S_j$. Specifically, we propose to choose the smallest component bound sequence, i.e., $S_j = \arg\min_{S_1, \dots, S_m} \abs{S_j}$. This minimizes the modifications we make to the pricing problem. In case there is no unique minimal component bound sequence, out of all those with minimal cardinality we then select the one where $K_j \coloneqq \sum_{q \in Q(S_j)} \lambda_q^*$ is closest to $\frac{Z}{2}$. Here, $Z$ denotes the number of aggregated subproblems in the current block (see Section \ref{sec:cg_bp_idsp} and Section \ref{sec:cmpbnd_separation_branching}). The intuition behind this is once again to maintain balance within the tree: if $K_j$ was far off from $\frac{Z}{2}$, i.e., either close to $0$ or close to $Z$, branching with $S_j$ would be little better than dichotomous branching, since it would either forbid almost all or almost no solutions.

\subsection{Post-processing of Component Bound Sequences}\label{sec:cmpbnd_separation_postprocessing}
Depending on the heuristics chosen, there is no guarantee that the separation procedure will find a component bound sequence $S$ in which each component $x_i$ has at most one upper bound (lower bound analogous). While this is not a problem from a mathematical standpoint, only the least upper bound (greatest lower bound, respectively) is relevant, and so adding variables and constraints for the other upper bounds (lower bounds) is unnecessary and could potentially slow down the solving process of \SP{}. Therefore, post-processing of the component bound sequences, i.e., removing redundant bounds, is advisable. For example, if we had $S = \{(x_1, \leq, 3), (x_1, \leq, 4)\}$, the first bound already implies the second, and we can remove the second bound from $S$, yielding $\{(x_1, \leq, 3)\}$.

\subsection{Branching with Multiple Subproblems}\label{sec:cmpbnd_separation_branching}
The component bound branching rule described above can be applied to instances with a single subproblem, as well as instances with multiple identical subproblems aggregated into a single subproblem (see Section \ref{sec:cg_bp_idsp}). However, there are instances consisting of at least two distinct (aggregated) subproblems, also known as blocks, where the master problem yields a solution ${\vec{\lambda}^k}^*_\RMP{}$ for each block $k$. Since each component $x_i$ belongs to a specific block, not all columns $q_1, q_2$ in \RMP{} will have an entry for $x_i$, thus the separation scheme is not directly applicable across multiple blocks.

Given that more than one block has fractional master solutions, we propose to pick one of those blocks to branch on and then apply the separation procedure as described above within the selected block.

\section{Comparison to Vanderbeck's Generic Branching}\label{sec:cmpbnd_simdif}
Both the generic branching scheme by Vanderbeck (Section \ref{sec:cg_bp_bp_branching_generic}) and the proposed component bound scheme involve branching in the master problem by imposing bounds on the original variables within the \SP{}. The primary difference lies in how these component bounds are imposed. \texttt{GENERIC} branching enforces these bounds as hard constraints, effectively subdividing the solution space of the original variables. Consequently, when the optimal solution $\vec{x}^*$ to the \IP{} is found in a branch-and-bound node of the \RMP{}, it satisfies all the component bounds imposed by the branching decisions from the root to that node. In contrast, our approach adds these bounds as soft constraints, allowing both columns that satisfy and those that violate the component bounds to be generated.

While the component bound branching rule might be easier to implement, branching using Vanderbeck's generic scheme has a significant advantage: It only requires tightening the bounds of the original variables in the \SP{}, whereas \texttt{COMPBND} branching introduces new variables and constraints. This is a significant advantage, since the structure of the pricing problem does not change. Many dynamic programming solvers for specific \IP{}s can handle changing variable bounds, and can therefore continue to generate columns efficiently, even after branching. This is unfortunately not the case for our approach, where the pricing problem changes with each branching decision, possibly forcing us to fall back to a generic \MIP{} solver. Furthermore, the \texttt{GENERIC} scheme tightens the bounds further as we descend deeper into the search tree, which in turn means that the pricing problem becomes increasingly easier to solve. In contrast, \texttt{COMPBND} branching will add more and more variables and constraints, which will further complicate the \SP{}.

\cleardoublepage
\chapter{Master Constraints without corresponding Original Problem Constraints}\label{ch:gm}
In addition to implementing the component bound branching rule (Chapter \ref{ch:cmpbnd}), a significant goal of this thesis is to enable future \GCG{} developers to easily create new branching rules and separators within the framework (Section \ref{sec:tools_gcg}). Currently, \GCG{} has limitations in this regard: branching rules must either produce decisions formulated in the original problem, which can then be Dantzig-Wolfe reformulated and added to the master and pricing problems (as seen when branching on original variables in Section \ref{sec:cg_bp_bp_branching_original}), or they must produce constraints for the master problem without requiring modifications to the pricing problem, as seen with Ryan-Foster branching. Other branching rules, such as Vanderbeck's generic branching scheme (Section \ref{sec:cg_bp_bp_branching_generic}), cannot be implemented without significant changes to the \GCG{} framework. These changes would involve applying and removing component bounds in the pricing problem when a node in the search tree is entered or left. Our proposed component bound branching rule (Chapter \ref{ch:cmpbnd}) and any separators using the master problem (Section \ref{sec:cg_bp_bpc_separators_master}) would also require such changes. The reason is that \GCG{} does not currently support imposing constraints in the master problem that necessitate modifications to at least one \SP{}, where the master constraints and induced pricing problem modifications cannot necessarily be described as a product of a Dantzig-Wolfe reformulation, i.e., do not necessarily have a counterpart in the original formulation.

In this chapter, we will specify the notation of such constraints, referred to as \textbf{generic mastercuts}. We will present our integration of these constraints into the \GCG{} framework as part of a new interface and demonstrate how to apply dual value stabilization to these constraints.

First, let us define the concept of a generic mastercut, which unites Vanderbeck's generic branching scheme, our component bound branching rule, and any master separators.

\begin{definition}\label{def:gm}
A \textbf{generic mastercut} is a constraint in the master problem that does not have a counterpart in the original problem, and therefore requires modification to one or multiple \SP{} to ensure its validity in the master. Specifically, it takes the following form, where the function $\operatorname{f}$ maps columns $p$ and $r$ to their respective coefficients in the master constraint:
\begin{equation*}
\sum_{p \in P} \operatorname{f}(p) \lambda_p + \sum_{r \in R} \operatorname{f}(r) \lambda_r \leq f \quad \left[\gamma\right]
\end{equation*}

The subproblems are now responsible for correctly determining the coefficients $\operatorname{f}(p)$ and $\operatorname{f}(p)$ of all newly generated columns $p$ and $r$. Therefore, one generic mastercut is associated with a set of pricing modifications, one for each subproblem that the constraint in the master affects.
\end{definition}

\begin{definition}\label{def:gm_pricing_modification}
A \textbf{pricing modification} to the subproblem $\SP{k}$ in block $k$, associated with a generic mastercut with dual value $\gamma$, is a set of constraints and variables added to the subproblem to ensure the validity of the generic mastercut in the master problem with respect to new variables.

Every pricing modification includes at least one mandatory variable $y \in Y$ of some domain $Y$ (e.g. $Y = \mathbb{Z}_+$) with an objective coefficient of $-\gamma$ in the $\SP{k}$. The solution value of $y$ is used as the column entry for the master constraint of the generic mastercut, i.e., $\operatorname{f}(p)$ or $\operatorname{f}(r)$. For this reason, variable is known as the \textbf{coefficient variable} of the pricing modification, and we modify the pricing problem as follows:

\begin{equation*}
\begin{aligned}
z^*_\SP{} = &\min & \left( \vec{c}\transpose - \vec{\pi}_\vec{b}\transpose \mat{A} \right) \vec{x} {\color{blue} - \gamma y} - \pi_0 & \\
&\st & \mat{D} \vec{x} & \geq \vec{d} \\
&& {\color{blue} y} & {\color{blue} = \operatorname{f}(\vec{x})} \\
&& \vec{x} & \in \mathbb{Z}_+^n \\
&& {\color{blue} y} & {\color{blue} \in Y}
\end{aligned}
\end{equation*}

Expressing $y = \operatorname{f}(\vec{x})$ may require auxiliary variables and constraints. Due to their auxiliary role, these variables have an objective coefficient of zero and do not correspond to a row in the master problem.
\end{definition}

This generic mastercut construct can be used by both the \texttt{GENERIC} (Section \ref{sec:cg_bp_bp_branching_generic}) and the \texttt{COMPBND} branching rule (Chapter \ref{ch:cmpbnd}): in both, we choose $\operatorname{f}$ to be an indicator function that equals one if and only if the column in question fully satisfies a given component bound sequence $S$. Thus, $y$ would be a binary decision variable ($Y = \{0, 1\}$).

In Vanderbeck's generic branching scheme, the pricing problems are only permitted to generate columns in the region defined by $S$. We enforce this, by creating auxiliary constraints $x_i \eta_i n_i$ for each $\left( x_{*,i}, \eta_i, n_i \right) \in S$, and forcing $y = 1 (= \operatorname{f}(\vec{x}))$ (or $Y = \{1\}$), though $y$ could be removed altogether through presolving. In contrast, \texttt{COMPBND} branching additionally allows columns violating $S$ to be generated, thus we create auxiliary variables and constraints to determine whether all bounds in $S$ were satisfied, as discussed in Section \ref{sec:cmpbnd_overview}.

The benefit of this construct is its generality, and thus its versatility. It does not presume anything about the origin of a generic mastercut. For this reason, it is not only applicable to branching rules, but, for example, also to separators. In Section \ref{sec:cg_bp_bpc_separators_master}, we have briefly discussed the possibility of separating a fractional master solution by finding cutting planes solely based on information of the Dantzig-Wolfe reformulation. Cuts found by such a master separator would exactly fit the definition of a generic mastercut.

In the following, we will see that a technical issue affecting the validity of the constraints arises when generic mastercuts are created locally in non-root nodes. We will briefly present how the current implementation of Vanderbeck's generic branching scheme in \GCG{} deals with this issue, and then introduce our solution that is correct, more efficient, and more broadly applicable. Finally, we will discuss how we can continue using dual value stabilization in \GCG{} with generic mastercuts.

\section{Mastervariable Synchronization across the entire Search Tree}\label{sec:gm_sync}
\begin{figure}[H]
\centering
\tikzset{every tree node/.style={minimum width=2em,draw,circle},
			blank/.style={draw=none},
			edge from parent/.style=
			{draw,edge from parent path={(\tikzparentnode) -- (\tikzchildnode)}},
			level distance=1.5cm, sibling distance=2.5em}
\begin{tikzpicture}
\Tree
[.A
	[.B
		[.D ]
		[.E ]
	]
	[.C
		[.F ]
		[.G ]
	]
]
\end{tikzpicture}
\caption{An exemplary search tree created by the component bound branching rule, where the lexicographic order of the nodes resembles the order in which they were created.}
\label{fig:gm_sync_tree}
\end{figure}

As we have mentioned multiple times, all columns in the \RMP{} must have the correct coefficient set in the master constraint. Just one column with an incorrect coefficient can lead to invalid mastercuts. For example, if a column $q$ satisfies a component bound sequence $S$, it should have a coefficient of $1$ in the mastercut. Any other coefficient would lead to an incomplete branching scheme. This very same reason is why the pricing modifications of a generic mastercut are essential in the first place.

So, we must ensure that all columns in the \RMP{} have the correct coefficients set. This can be easily achieved when creating the generic mastercut, as well as when a new column is generated in the subtree of the node where the generic mastercut was created. In the former case, we may simply compute the coefficients of all variables in the \RMP{} upfront. And in the later case, the solution value of the coefficient variable in the \SP{} already determines the correct coefficient in the master. However, consider the following scenario in a search tree, for example a tree generated with the component bound branching rule using generic mastercuts, as depicted in Figure \ref{fig:gm_sync_tree}: we are currently processing node F in the search tree and generate a new column $q'$. After deactivating node F and activating node D, it is possible that $x_{q'}$ satisfies the component bounds imposed in D. Thus, $q'$ should have a coefficient of $1$ in the mastercut of D. However, since the column was generated in F, the information that $q'$ was created was not communicated to D. And therefore, the coefficient of $q'$ in the mastercut of D is not set correctly. This problem also occurs for cuts produced by master separators.

To prevent this, all generic mastercuts must be made aware of these columns to update their coefficients accordingly. Specifically, we want to synchronize newly generated master variables across the entire search tree lazily, i.e., only when a node is activated and thus the update is required. Moreover, since \GCG{} can remove columns it deems unnecessary from the \MP{}, the synchronization must consider the case where a newly generated column is deleted before it is fully synchronized across the search tree.

In this section, we will first analyze the current approach taken by the implementation of Vanderbeck's generic branching in \GCG{}. Then, we will present a more efficient approach, which we refer to as \textbf{history tracking}, and further optimize it.

\subsection{Current Approach used by the Implementation of Vanderbeck's Generic Branching}\label{subsec:gm_sync_current}
In the current implementation of Vanderbeck's generic branching in \GCG{}, each node created by this branching rule stores the number of master variables it is aware of. This number is updated whenever a node is deactivated to reflect any newly generated columns. Upon node activation, the current number of variables in the master is compared against how many variables were present the last time the node was active. If new columns have been added to the \RMP{} in the meantime, this counter might increase. If so, the coefficient for the new columns will be determined and set in the \RMP{}. More specifically, since the number of variables in the master can grow quite large, it avoids updating the coefficients of all columns in the \RMP{}. Instead, it assumes the master variables indexed from the last known number of variables to the current number of variables are new and sets their coefficients accordingly.

Unfortunately, for any generic mastercut in general, this approach is not sufficient, as not all new columns are necessarily detected. For instance, if one column was generated and another column was deleted in the meantime, the counter would not increase, as the number of columns in the master would remain the same. Consequently, the coefficient for the new column would not be set in the mastercut, potentially leading to invalid mastercuts.

Additionally, this approach is not developer-friendly. Developers of branching rules and master separators should not have to manage when and where columns are generated and deleted. This is a task that should be handled by \GCG{}, abstracting away the origins of the columns.

\subsection{History Tracking Approach}\label{subsec:gm_sync_history}
\begin{figure}[H]
\centering
\begin{tikzpicture}[historynode/.style={
						draw,
						rectangle split,
						rectangle split horizontal,
						rectangle split parts=2,
					},
					every edge/.style={draw, -{Stealth[length=2mm, width=2mm]}},
					node distance=1cm and 1cm]

% Nodes
\node (node0) [historynode] at (0,0) {\nodepart{one} $q_0$ \nodepart{two} $1$};
\node (node1) [historynode,right=of node0] {\nodepart{one} $q_1$ \nodepart{two} $1$};
\node (node2) [historynode,right=of node1] {\nodepart{one} $q_2$ \nodepart{two} $3$};
\node (node3) [historynode,right=of node2] {\nodepart{one} $q_3$ \nodepart{two} $2$};
\node (node4) [historynode,right=of node3] {\nodepart{one} $q_4$ \nodepart{two} $1$};
\node (node5) [historynode,right=of node4] {\nodepart{one} $q_5$ \nodepart{two} $2$};

% Edges
\draw[->] (node0.two east) -- (node1.one west);
\draw[->] (node1.two east) -- (node2.one west);
\draw[->] (node2.two east) -- (node3.one west);
\draw[->] (node3.two east) -- (node4.one west);
\draw[->] (node4.two east) -- (node5.one west);

% External references
\draw[->,dashed] (node0.one split south) ++(0,-1) node[below,draw=none] {C} -- ++(0,1);
\draw[->,dashed] (node2.one split south) ++(-0.3,-1) node[below,draw=none] {A} -- ++(0.2,1);
\draw[->,dashed] (node2.one split south) ++(0.3,-1) node[below,draw=none] {B} -- ++(-0.2,1);
\draw[->,dashed] (node3.one split south) ++(0,-1) node[below,draw=none] {\textit{other}} -- ++(0,1);
\draw[->,dashed] (node5.one split south) ++(0,-1) node[below,draw=none] {\texttt{Latest}} -- ++(0,1);

\end{tikzpicture}
\caption{Reference-counted linked list of the history of columns added to the \RMP{}, with external references drawn dashed from below, e.g., those from the search tree nodes A, B, and C. Each element holds a reference to the master variable belonging to column $q_i$, as well as the number of references to itself.}
\label{fig:gm_sync_history}
\end{figure}

We propose an efficient approach to lazily notify all nodes in the search tree upon node activation of new columns generated while also considering deleted columns. We introduce a reference-counted linked list of variables added to the \RMP{}, where the order of the variables in the list is determined by their generation order. Each node in the search tree holds its own external reference to this construct. The specific element in the list that a search tree node points to indicates the last column in the \RMP{} when the node was last active. All subsequent variables, i.e., the elements next in the list, are new columns generated elsewhere in the tree. Additionally, we hold one external reference to the tail of the list, representing the last generated column. This construct is illustrated in Figure \ref{fig:gm_sync_history}. Since the linked list tracks which variables were created when we will refer to this list as the \texttt{varhistory}.

Let us consider a search tree with root node A and child nodes B and C. Currently, we are solving node B, and therefore nodes A and B are active. While solving B, we have already generated columns $q_3$, $q_4$, and $q_5$. Assume we have solved the relaxation of B to optimality, finding a fractional solution, and have created two child nodes D and E. Both nodes will be created using all columns currently in the \RMP{}, i.e., $q_i, i \in \{0, 5\}$. For this reason, we use the \texttt{Latest} pointer to initialize the reference to the \texttt{varhistory} of D and E (Figure \ref{fig:gm_sync_history_d_e}).

\begin{figure}[H]
\centering
\begin{tikzpicture}[historynode/.style={
						draw,
						rectangle split,
						rectangle split horizontal,
						rectangle split parts=2,
					},
					every edge/.style={draw, -{Stealth[length=2mm, width=2mm]}},
					node distance=1cm and 1cm]

% Nodes
\node (node0) [historynode] at (0,0) {\nodepart{one} $q_0$ \nodepart{two} $1$};
\node (node1) [historynode,right=of node0] {\nodepart{one} $q_1$ \nodepart{two} $1$};
\node (node2) [historynode,right=of node1] {\nodepart{one} $q_2$ \nodepart{two} $3$};
\node (node3) [historynode,right=of node2] {\nodepart{one} $q_3$ \nodepart{two} $2$};
\node (node4) [historynode,right=of node3] {\nodepart{one} $q_4$ \nodepart{two} $1$};
\node (node5) [historynode,right=of node4] {\nodepart{one} $q_5$ \nodepart{two} $4$};

% Edges
\draw[->] (node0.two east) -- (node1.one west);
\draw[->] (node1.two east) -- (node2.one west);
\draw[->] (node2.two east) -- (node3.one west);
\draw[->] (node3.two east) -- (node4.one west);
\draw[->] (node4.two east) -- (node5.one west);

% External references
\draw[->,dashed] (node0.one split south) ++(0,-1) node[below,draw=none] {C} -- ++(0,1);
\draw[->,dashed] (node2.one split south) ++(-0.3,-1) node[below,draw=none] {A} -- ++(0.2,1);
\draw[->,dashed] (node2.one split south) ++(0.3,-1) node[below,draw=none] {B} -- ++(-0.2,1);
\draw[->,dashed] (node3.one split south) ++(0,-1) node[below,draw=none] {\textit{other}} -- ++(0,1);
\draw[->,dashed] (node5.one split south) ++(-1,-1) node[below,draw=none] {D} -- ++(0.7,1);
\draw[->,dashed] (node5.one split south) ++(-0.4,-1) node[below,draw=none] {E} -- ++(0.35,1);
\draw[->,dashed] (node5.one split south) ++(0.6,-1) node[below,draw=none] {\texttt{Latest}} -- ++(-0.5,1);

\end{tikzpicture}
\caption{\texttt{varhistory} after creating child nodes D and E of node B.}
\label{fig:gm_sync_history_d_e}
\end{figure}

Continuing this scenario, let \GCG{} deem the column $q_2$ unnecessary and remove it from the \RMP{}. Since there may be external references to this variable, which in this case there are, we do not remove the element in the list holding $q_2$. Instead, we mark it as deleted. Next, we would like to solve node C. For this, we must deactivate node B, and activate node C. Whenever we deactivate a node, we know that it and all its ancestors are already aware of all columns in the \RMP{}. Therefore, we may jump all the active node's pointers to the \texttt{Latest} pointer. This is illustrated in Figure \ref{fig:gm_sync_history_deactivate_B}.

\begin{figure}[H]
\centering
\begin{tikzpicture}[historynode/.style={
						draw,
						rectangle split,
						rectangle split horizontal,
						rectangle split parts=2,
					},
					every edge/.style={draw, -{Stealth[length=2mm, width=2mm]}},
					node distance=1cm and 1cm]

% Nodes
\node (node0) [historynode] at (0,0) {\nodepart{one} $q_0$ \nodepart{two} $1$};
\node (node1) [historynode,right=of node0] {\nodepart{one} $q_1$ \nodepart{two} $1$};
\node (node2) [historynode,right=of node1] {\nodepart{one} $\stkout{q_2}$ \nodepart{two} $1$};
\node (node3) [historynode,right=of node2] {\nodepart{one} $q_3$ \nodepart{two} $2$};
\node (node4) [historynode,right=of node3] {\nodepart{one} $q_4$ \nodepart{two} $1$};
\node (node5) [historynode,right=of node4] {\nodepart{one} $q_5$ \nodepart{two} $6$};

% Edges
\draw[->] (node0.two east) -- (node1.one west);
\draw[->] (node1.two east) -- (node2.one west);
\draw[->] (node2.two east) -- (node3.one west);
\draw[->] (node3.two east) -- (node4.one west);
\draw[->] (node4.two east) -- (node5.one west);

% External references
\draw[->,dashed] (node0.one split south) ++(0,-1) node[below,draw=none] {C} -- ++(0,1);
\draw[->,dashed] (node3.one split south) ++(0,-1) node[below,draw=none] {\textit{other}} -- ++(0,1);
\draw[->,dashed] (node5.one split south) ++(-1.3,-1) node[below,draw=none] {A} -- ++(1.1,1);
\draw[->,dashed] (node5.one split south) ++(-0.7,-1) node[below,draw=none] {B} -- ++(0.6,1);
\draw[->,dashed] (node5.one split south) ++(-0.1,-1) node[below,draw=none] {D} -- ++(0.05,1);
\draw[->,dashed] (node5.one split south) ++(0.5,-1) node[below,draw=none] {E} -- ++(-0.4,1);
\draw[->,dashed] (node5.one split south) ++(1.5,-1) node[below,draw=none] {\texttt{Latest}} -- ++(-1.2,1);

\end{tikzpicture}
\caption{\texttt{varhistory} after deletion of column $q_2$ and deactivation of node B.}
\label{fig:gm_sync_history_deactivate_B}
\end{figure}

Finally, we can activate node C. Upon node activation, we realize that the element that C points to in the \texttt{varhistory} has a next element. This means that there are new columns that have been generated since the last time C was active. We forward the pointer of C one by one until we reach the \texttt{Latest} pointer. Each time we forward the pointer, if the variable $q_i$ has not been marked as deleted, we calculate the coefficient of $q_i$ in the generic mastercut of C.

Whenever we forward a pointer, either step-by-step or by jumping to the \texttt{Latest} pointer, the internal reference count of the elements in the list is updated. As soon as this reference count reaches zero, the element will be safely freed. This ensures that only necessary variables, i.e., those that still need to be synchronized across the entire search tree, are kept in memory. This is illustrated in Figure \ref{fig:gm_sync_history_activate_C}.

\begin{figure}[H]
\centering
\begin{tikzpicture}[historynode/.style={
						draw,
						rectangle split,
						rectangle split horizontal,
						rectangle split parts=2,
					},
					every edge/.style={draw, -{Stealth[length=2mm, width=2mm]}},
					node distance=1cm and 1cm]

% Nodes
\node (node3) [historynode] {\nodepart{one} $q_3$ \nodepart{two} $1$};
\node (node4) [historynode,right=of node3] {\nodepart{one} $q_4$ \nodepart{two} $1$};
\node (node5) [historynode,right=of node4] {\nodepart{one} $q_5$ \nodepart{two} $7$};

% Edges
\draw[->] (node3.two east) -- (node4.one west);
\draw[->] (node4.two east) -- (node5.one west);

% External references
\draw[->,dashed] (node3.one split south) ++(0,-1) node[below,draw=none] {\textit{other}} -- ++(0,1);
\draw[->,dashed] (node5.one split south) ++(-1.6,-1) node[below,draw=none] {A} -- ++(1.3,1);
\draw[->,dashed] (node5.one split south) ++(-1,-1) node[below,draw=none] {B} -- ++(0.8,1);
\draw[->,dashed] (node5.one split south) ++(-0.4,-1) node[below,draw=none] {C} -- ++(0.3,1);
\draw[->,dashed] (node5.one split south) ++(0.2,-1) node[below,draw=none] {D} -- ++(-0.1,1);
\draw[->,dashed] (node5.one split south) ++(0.8,-1) node[below,draw=none] {E} -- ++(-0.6,1);
\draw[->,dashed] (node5.one split south) ++(1.8,-1) node[below,draw=none] {\texttt{Latest}} -- ++(-1.5,1);

\end{tikzpicture}
\caption{\texttt{varhistory} after activation of node C.}
\label{fig:gm_sync_history_activate_C}
\end{figure}

Assume node C generates a new column $q_6$. We add this column to the \texttt{varhistory} by allocating a new list element, setting its reference count to $1$, setting the next pointer of the current \texttt{Latest} element to the new element, and finally forwarding the \texttt{Latest} pointer to the new element. This is illustrated in Figure \ref{fig:gm_sync_history_generate_q6}.

\begin{figure}[H]
\centering
\begin{tikzpicture}[historynode/.style={
						draw,
						rectangle split,
						rectangle split horizontal,
						rectangle split parts=2,
					},
					every edge/.style={draw, -{Stealth[length=2mm, width=2mm]}},
					node distance=1cm and 1cm]

% Nodes
\node (node3) [historynode] {\nodepart{one} $q_3$ \nodepart{two} $1$};
\node (node4) [historynode,right=of node3] {\nodepart{one} $q_4$ \nodepart{two} $1$};
\node (node5) [historynode,right=of node4] {\nodepart{one} $q_5$ \nodepart{two} $6$};
\node (node6) [historynode,right=of node5] {\nodepart{one} $q_6$ \nodepart{two} $2$};

% Edges
\draw[->] (node3.two east) -- (node4.one west);
\draw[->] (node4.two east) -- (node5.one west);
\draw[->] (node5.two east) -- (node6.one west);

% External references
\draw[->,dashed] (node3.one split south) ++(0,-1) node[below,draw=none] {\textit{other}} -- ++(0,1);
\draw[->,dashed] (node5.one split south) ++(-1.6,-1) node[below,draw=none] {A} -- ++(1.3,1);
\draw[->,dashed] (node5.one split south) ++(-1,-1) node[below,draw=none] {B} -- ++(0.8,1);
\draw[->,dashed] (node5.one split south) ++(-0.4,-1) node[below,draw=none] {C} -- ++(0.3,1);
\draw[->,dashed] (node5.one split south) ++(0.2,-1) node[below,draw=none] {D} -- ++(-0.1,1);
\draw[->,dashed] (node5.one split south) ++(0.8,-1) node[below,draw=none] {E} -- ++(-0.6,1);
\draw[->,dashed] (node6.one split south) ++(0,-1) node[below,draw=none] {\texttt{Latest}} -- ++(0,1);

\end{tikzpicture}
\caption{\texttt{varhistory} after generating column $q_6$ in node C.}
\label{fig:gm_sync_history_generate_q6}
\end{figure}

This approach is correct in the sense that all active generic mastercuts are guaranteed to be aware of all columns in the \RMP{}. This correctness is given, since the deletion of variables cannot not cause us to miss any new variables. Since close to no management is required, its performance impact is negligible: we must only update reference counts, forward pointers, and append and free list elements. Furthermore, we hold the memory footprint to a minimum, as the \texttt{varhistory} construct will only hold variables that still need to be synchronized. But once all external references have seen some variable $q_i$, i.e., its reference count reaches zero, we can automatically free the memory of the element in the list holding $q_i$.

And as a final note, this approach is not limited for synchronization of master variables for generic mastercuts used as branching decisions, but can also be used for other purposes, which have symbolized by the \textit{other} reference in the above figures. Such other purposes, for example would be keeping cutting planes generated by master separators up-to-date (Section \ref{sec:cg_bp_bpc_separators_master}).

\subsection{History Tracking using Unrolled Linked Lists Approach}\label{subsec:gm_sync_history_unrolled}
While already efficient, the previous approach has an opportunity for improvement: the elements of the \texttt{varhistory} might be allocated in completely different memory locations, leading to poor cache utilization during traversal. Storing the entire list in a contiguous memory block could improve cache locality but could be costly if reallocation and copying are needed when space runs out.

We can improve cache utilization by unrolling the linked list into blocks storing a fixed number of columns, with each block having its own reference count. These blocks are linked together, forming a list of blocks. The references to the \texttt{varhistory} point to the blocks and hold an offset within the block. This way, each reference still refers to some unique column $q_i$, retaining the ability to forward a pointer one column at a time. A new variable is added to the tail block if its capacity isn't maxed out; otherwise, a new block is allocated. This concept is illustrated in Figure \ref{fig:gm_sync_history_unrolled}.

\begin{figure}[H]
\centering
\begin{tikzpicture}[historynode/.style={
						draw,
						rectangle split,
						rectangle split horizontal,
						rectangle split parts=5,
					},
					every edge/.style={draw, -{Stealth[length=2mm, width=2mm]}},
					node distance=1cm and 1cm]

% Nodes
\node (node0) [historynode] at (0,0) {
	\nodepart{one} $q_0$
	\nodepart{two} $q_3$
	\nodepart{three} $q_2$
	\nodepart{four} $q_3$
	\nodepart{five} $4$
};
\node (node1) [historynode,right=of node0] {
	\nodepart{one} $q_4$
	\nodepart{two} $q_5$
	\nodepart{three}
	\nodepart{four}
	\nodepart{five} $2$
};

% Edges
\draw[->] (node0.two east) -- (node1.one west);

% External references
\draw[->,dashed] (node0.three south) ++(-1.8,-1) node[below,draw=none] {C,0} -- ++(1.6,1);
\draw[->,dashed] (node0.three south) ++(-0.9,-1) node[below,draw=none] {A,2} -- ++(0.8,1);
\draw[->,dashed] (node0.three south) ++(0,-1) node[below,draw=none] {B,2} -- ++(0,1);
\draw[->,dashed] (node0.three south) ++(1.2,-1) node[below,draw=none] {\textit{other},3} -- ++(-1.1,1);
\draw[->,dashed] (node1.three south) ++(0,-1) node[below,draw=none] {\texttt{Latest},1} -- ++(0,1);

\end{tikzpicture}
\caption{\texttt{varhistory} of Figure \ref{fig:gm_sync_history} unrolled into blocks of size 4.}
\label{fig:gm_sync_history_unrolled}
\end{figure}

This approach improves cache locality and reduces the total number of memory allocation and deallocation operations. From an outside perspective, the fundamental operations of the \texttt{varhistory} construct remain the same.

\section{Dual Value Stabilization for Generic Mastercuts}\label{sec:gm_dvs}
\cleardoublepage
\chapter{Implementation}\label{ch:implementation}

\section{Generic Mastercuts}\label{sec:implementation_gm}
\section{Mastervariable Synchronization}\label{sec:implementation_sync}
As discussed in Section \ref{sec:gm_sync}, there is a need to synchronize the information of newly generated columns across the entire search tree. This requirement arose in the context of branching using generic mastercuts, but it is generally necessary for any branching rule that does not formulate its decisions in the original problem. Therefore, we have decoupled the synchronization of master variables from the generic mastercut interface and implemented it as a separate internal module within \GCG{}.

\begin{lstlisting}[language=C, caption=Variable History Construct]
struct GCG_VARHISTORYBUFFER {
	SCIP_VAR*             vars[50];
	int                   nvars;
	GCG_VARHISTORYBUFFER* next;
	int                   nuses;
};

struct GCG_VARHISTORY {
	GCG_VARHISTORYBUFFER* buffer;
	int                   pos;
};
\end{lstlisting}

To enable such mastervariable synchronization, we have implemented the history tracking approach using unrolled linked lists as described in Section \ref{subsec:gm_sync_history_unrolled}. Each element, or buffer, in the unrolled linked list has a default capacity of 50 variables. The variable \texttt{nuses}, which acts as a reference count, keeps track of how many strong pointers of type \texttt{GCG\_VARHISTORY} are currently pointing to the buffer. The \texttt{GCG\_VARHISTORY} structure is a simple wrapper around the buffer, keeping track of the current position in the buffer. Consequently, each strong pointer still points to a specific variable. The \GCG{} pricer is responsible for managing the global variable history list and appending new variables whenever a new column is generated. In Section \ref{sec:gm_sync}, we have denoted this central reference as the \texttt{Latest} pointer.

We can now synchronize master variables by attaching a strong reference to each node in the search tree. Specifically, upon node creation, we create a reference identical to the \texttt{Latest} pointer stored in the \GCG{} pricer. Then, following the procedure described in Section \ref{subsec:gm_sync_history}, we forward these strong pointers upon node (de-)activation. To inform a branching rule in a specific node about the creation of a new variable, we have extended the \GCG{} branching rule interface with the following callback function. The branching rule can then, for example, determine the constraint coefficient for this new master variable.

\begin{lstlisting}[language=C, caption=Branching Rule Interface Extension]
#define GCG_DECL_BRANCHNEWCOL(x) SCIP_RETCODE x (SCIP* scip, GCG_BRANCHDATA* branchdata, SCIP_VAR* mastervar)
\end{lstlisting}

Finally, we note that each \texttt{SCIP\_VAR} already contains a flag indicating whether it has been deleted from the problem. We use this flag instead of marking such variables as deleted in the history buffer ourselves.

This variable history is also used by Reinartz Groba \cite{reinartzgroba2024todo} to keep master cuts up to date.

\section{Component Bound Branching}\label{sec:cmpbnd}
\cleardoublepage
\chapter{Evaluation}\label{ch:evaluation}

\section{Testset of Instances}\label{sec:evaluation_testset}
\begin{figure}
	\centering

	\begin{subfigure}{0.495\textwidth}
		\centering
		\includesvg[width=\textwidth]{images/general/MostFractional/solve_status.svg}
		\caption{\texttt{MostFractional}: solve status}
		\label{fig:mostfractional_solve_status}
	\end{subfigure}
	\hfill
	\begin{subfigure}{0.495\textwidth}
		\centering
		\includesvg[width=\textwidth]{images/general/ClosestToZHalf/solve_status.svg}
		\caption{\texttt{ClosestToZHalf}: solve status}
		\label{fig:closesttozhalf_solve_status}
	\end{subfigure}

	\vspace{1em}

	\begin{subfigure}{0.495\textwidth}
		\centering
		\includesvg[width=\textwidth]{images/general/MostFractional/nodes.svg}
		\caption{\texttt{MostFractional}: nodes}
		\label{fig:mostfractional_nodes}
	\end{subfigure}
	\hfill
	\begin{subfigure}{0.495\textwidth}
		\centering
		\includesvg[width=\textwidth]{images/general/ClosestToZHalf/nodes.svg}
		\caption{\texttt{ClosestToZHalf}: nodes}
		\label{fig:closesttozhalf_nodes}
	\end{subfigure}

	\vspace{1em}

	\begin{subfigure}{0.495\textwidth}
		\centering
		\includesvg[width=\textwidth]{images/general/MostFractional/times.svg}
		\caption{\texttt{MostFractional}: times}
		\label{fig:mostfractional_times}
	\end{subfigure}
	\hfill
	\begin{subfigure}{0.495\textwidth}
		\centering
		\includesvg[width=\textwidth]{images/general/ClosestToZHalf/times.svg}
		\caption{\texttt{ClosestToZHalf}: times}
		\label{fig:closesttozhalf_times}
	\end{subfigure}

	\caption{Comparison of all run configurations. In the boxplots the dots represent the arithmetic mean of the data. Outliers are not visualized. Note the differing scales in Figures \ref{fig:mostfractional_times} and \ref{fig:closesttozhalf_times}.}
	\label{fig:comparison_general}
\end{figure}

\section{Comparison of the different Separation Heuristics}\label{sec:evaluation_comparison_separation}
We compared different separation heuristics with and without full-tree dual value stabilization by running the test set on 12 configurations of the component bound branching rule. These configurations varied the first- and second-stage separation heuristics (Section \ref{sec:cmpbnd_separation}) and the stabilization method. The naming conventions for these runs are as follows: runs using both the \texttt{MaxRangeMidrange} and \texttt{MostDistinctMedian} first-stage heuristics are named \texttt{compbnd}. Runs using only \texttt{MaxRangeMidrange} are named \texttt{compbnd-mrm}, and those using only the \texttt{MostDistinctMedian} heuristic are named \texttt{compbnd-mdm}. If full-tree dual value stabilization was applied, \texttt{+dvs} is appended to the name. For example, \texttt{compbnd-mdm+dvs} uses the \texttt{MostDistinctMedian} heuristic with full-tree dual value stabilization.

We also proposed two options for the second-stage heuristic: \texttt{ClosestToZHalf} and \texttt{MostFractional}. For readability, we grouped all runs by their second-stage heuristic and presented their statistics in separate figures.

For reference, we included Vanderbeck's generic branching scheme results, denoted as \texttt{generic}. This allows us to compare the component bound branching rule's performance against generic branching scheme. As discussed in Section \ref{sec:cmpbnd_simdif}, we expect the generic branching scheme to outperform the component bound branching rule, partially due to the latter having to fall back to a general \MIP{} solver, while the former may retain the ability to use special-case solving algorithms after branching. To measure the impact of having to resort to a \MIP{} solver, we also included a configuration of Vanderbeck's scheme in which we force the use of a \MIP{} solver at all non-root nodes, denoted as \texttt{generic-mip}.

Analyzing the results in Figure \ref{fig:comparison_general}, several key observations emerge. First, full-tree dual value stabilization generally degrades the performance of the component bound branching rule. Second, runs using the \texttt{MostFractional} heuristic significantly outperform those using the \texttt{ClosestToZHalf} heuristic.

Among the best-performing configurations, i.e., those using the \texttt{MostFractional} second-stage heuristic, the number of instances solved and the solving times significantly improve when using both first-stage heuristics instead of just one. This pattern suggests that neither first-stage heuristic universally finds the optimal component bound sequences, highlighting the importance of the second-stage heuristic in selecting the best branching decisions.

\begin{figure}
	\centering

	\begin{subfigure}{0.495\textwidth}
		\centering
		\includesvg[width=\textwidth]{images/general/MostFractional/outperforms_generic.svg}
	\end{subfigure}
	\hfill
	\begin{subfigure}{0.495\textwidth}
		\centering
		\includesvg[width=\textwidth]{images/general/ClosestToZHalf/outperforms_generic.svg}
	\end{subfigure}

	\caption{Percentage of instances solved faster by the configurations of the component bound branching rule compared to \texttt{generic}.}
	\label{fig:comparison_outperform}
\end{figure}

\section{Comparison to Vanderbeck's Generic Branching}\label{sec:evaluation_comparison_generic}
As discussed in Section \ref{sec:cmpbnd_simdif}, the main difference between Vanderbeck's generic scheme and the component bound branching rule is in their modifications to the pricing problem. The \texttt{GENERIC} rule retains the pricing structure, allowing the use of specialized algorithms (e.g., knapsack solvers) at all nodes. In contrast, the \texttt{COMPBND} rule adds variables and constraints to the pricing problem, often necessitating a fallback to a general \MIP{} solver, which may degrade performance. Additionally, as the search tree deepens, the \texttt{COMPBND} rule further complicates the pricing problem by adding more variables and constraints, whereas the \texttt{GENERIC} rule's pricing problems become easier to solve due to tighter bounds. Thus, we expected the \texttt{GENERIC} rule to outperform the \texttt{COMPBND} rule, particularly for larger instances.

This expectation is confirmed by our results (Figure \ref{fig:comparison_general}). Vanderbeck's generic branching scheme solves more instances and does so in significantly less time compared to any configuration of the component bound branching rule.

Figure \ref{fig:comparison_outperform} shows how often the component bound branching rule outperforms Vanderbeck's generic branching scheme. The \texttt{MostFractional} second-stage heuristic consistently outperforms the \texttt{ClosestToZHalf} heuristic. Notably, the highest outperformance rate is achieved when only using the \texttt{MostDistinctMedian} first-stage heuristic. Combining it with the \texttt{MaxRangeMidrange} heuristic actually decreases the outperformance rate, suggesting that \texttt{MostDistinctMedian} is the most effective first-stage heuristic, while \texttt{MaxRangeMidrange} may not be as beneficial.

The surprisingly large outperformance rates we see for the \texttt{MostFractional} second-stage heuristic should be taken with a grain of salt. When we take a closer look at those instances, we observe that most of them are solved within 40 seconds by either branching rule, and often only a few seconds were saved. Therefore, the impact of this outperformance is limited. Especially given that the generic scheme is generally faster and solves more instances, it remains the preferred choice for most instances.

\begin{figure}
	\centering

	\begin{subfigure}{0.495\textwidth}
		\centering
		\includesvg[width=\textwidth]{images/general/MostFractional/outperforms_generic-mip.svg}
	\end{subfigure}
	\hfill
	\begin{subfigure}{0.495\textwidth}
		\centering
		\includesvg[width=\textwidth]{images/general/ClosestToZHalf/outperforms_generic-mip.svg}
	\end{subfigure}

	\caption{Percentage of instances solved faster by the configurations of the component bound branching rule compared to \texttt{generic-mip}.}
	\label{fig:comparison_outperform_generic_mip}
\end{figure}

Things change quite a bit for Vanderbeck's generic branching when we enforce the use of a \MIP{} solver for the pricing problem in non-root nodes. Although this \texttt{generic-mip} configuration produces fewer nodes, possibly due to more suitable columns being generated, having to rely on a \MIP{} solver imposes a great performance penalty, see Figure \ref{fig:comparison_general}. It seems as if creating a smaller search tree is far more beneficial than having to solve more complex pricing problems. Even the configurations of the weaker \texttt{ClosestToZHalf} second-stage heuristic seemingly perform on par with the generic branching scheme when it is forced to use a \MIP{} solver for pricing. The \texttt{MostFractional} second-stage heuristic configurations clearly outperform the generic branching scheme in this scenario, as shown in Figure \ref{fig:comparison_outperform_generic_mip}.

These findings suggest that the component bound branching rule can be a viable alternative to Vanderbeck's branching scheme when no specialized algorithms are available for the pricing problem. However, the generic branching scheme remains the preferred choice when such algorithms are available.

\section{In-Depth Analysis of the First-Stage Separation Heuristics and the Effect of Dual Value stabilization}\label{sec:evaluation_comparison_separation_firststage}
We now examine the first-stage separation heuristics in more detail, focusing on the selected component bound sequences for branching. Since the \texttt{MostFractional} second-stage heuristic significantly outperforms the \texttt{ClosestToZHalf} heuristic, we limit our analysis to configurations using the former. For each configuration and all instances, we logged the size of the component bound sequences at each node.

Although the \texttt{compbnd} configuration always selects the minimal component bound sequence from the two first-stage heuristics, this does not mean it branches with fewer component bounds on average compared to the \texttt{compbnd-mrm} or \texttt{compbnd-mdm} configurations. Current branching decisions influence future opportunities, and mixing both first-stage heuristics can lead to more component bounds per branching decision.

\begin{figure}
	\centering

	\begin{subfigure}{0.495\textwidth}
		\centering
		\includesvg[width=\textwidth]{images/bound_stats/num_bounds.svg}
		\caption{Distribution of the number of bounds created while branching}
		\label{fig:compbnd_num_bounds}
	\end{subfigure}
	\hfill
	\begin{subfigure}{0.495\textwidth}
		\centering
		\includesvg[width=\textwidth]{images/bound_stats/avg_num_bounds.svg}
		\caption{Average number of bounds created while branching}
		\label{fig:compbnd_avg_num_bounds}
	\end{subfigure}

	\caption{Left: distribution of the number of bounds created at each node for the different configurations. Note the logarithmic scale. Right: average number of bounds created overall for the different configuration.}
	\label{fig:comparison_bounds}
\end{figure}

Figure \ref{fig:compbnd_num_bounds} shows the distribution of the number of component bounds created for each branching decision. The data indicates that it is rare to branch with a component bound sequence larger than 4. Most cases involve sequences of size 1 or 2, as seen in Figure \ref{fig:compbnd_avg_num_bounds}. Additionally, both first-stage heuristics individually create more bounds on average than their combination, likely explaining why the \texttt{compbnd} configuration outperforms the others (Section \ref{sec:evaluation_comparison_separation}).

\begin{figure}
	\centering

	\begin{subfigure}{0.495\textwidth}
		\centering
		\includesvg[width=\textwidth]{images/bound_stats/compbnd/depths_distribution.svg}
		\caption{\texttt{compbnd(+dvs)}: Number of branching decisions per depth}
		\label{fig:compbnd_depths_distribution}
	\end{subfigure}
	\hfill
	\begin{subfigure}{0.495\textwidth}
		\centering
		\includesvg[width=\textwidth]{images/bound_stats/compbnd/avg_num_bounds_vs_depth.svg}
		\caption{\texttt{compbnd(+dvs)}: Average number of bounds per depth (95\% CI)}
		\label{fig:compbnd_avg_num_bounds_vs_depth}
	\end{subfigure}

	\vspace{1em}

	\begin{subfigure}{0.495\textwidth}
		\centering
		\includesvg[width=\textwidth]{images/bound_stats/compbnd-mdm/depths_distribution.svg}
		\caption{\texttt{compbnd-mdm(+dvs)}: Number of branching decisions per depth}
		\label{fig:compbnd-mdm_depths_distribution}
	\end{subfigure}
	\hfill
	\begin{subfigure}{0.495\textwidth}
		\centering
		\includesvg[width=\textwidth]{images/bound_stats/compbnd-mdm/avg_num_bounds_vs_depth.svg}
		\caption{\texttt{compbnd-mdm(+dvs)}: Average number of bounds per depth (95\% CI)}
		\label{fig:compbnd-mdm_avg_num_bounds_vs_depth}
	\end{subfigure}

	\vspace{1em}

	\begin{subfigure}{0.495\textwidth}
		\centering
		\includesvg[width=\textwidth]{images/bound_stats/compbnd-mrm/depths_distribution.svg}
		\caption{\texttt{compbnd-mrm(+dvs)}: Number of branching decisions per depth}
		\label{fig:compbnd-mrm_depths_distribution}
	\end{subfigure}
	\hfill
	\begin{subfigure}{0.495\textwidth}
		\centering
		\includesvg[width=\textwidth]{images/bound_stats/compbnd-mrm/avg_num_bounds_vs_depth.svg}
		\caption{\texttt{compbnd-mrm(+dvs)}: Average number of bounds per depth (95\% CI)}
		\label{fig:compbnd-mrm_avg_num_bounds_vs_depth}
	\end{subfigure}

	\caption{Left: number of branching decisions per depth. Right: average number of component bounds per depth, with 95\% confidence interval (smoothed).}
	\label{fig:comparison_depth}
\end{figure}

Figure \ref{fig:comparison_depth} illustrates the number of branching decisions per depth and the average number of component bounds per branching decision across different depths. Given that the \texttt{COMPBND} rule creates a binary search tree and each instance is solved with only a few hundred nodes (Figure \ref{fig:mostfractional_nodes}), most branching decisions occur at depths in the low hundreds. Consequently, we plotted the average number of component bounds per branching decision up to depth 500.

Our first observation is that full-tree dual value stabilization has little effect on the average number of component bounds per branching decision. Since each configuration produces a similar number of nodes with and without stabilization (Figure \ref{fig:mostfractional_nodes}), we further estimate that both search trees are similar. Therefore, as roughly the same amount of nodes are solved with similar complexities of the pricing problems, the performance degradation discussed in Section \ref{sec:evaluation_comparison_separation} suggests that the management cost of dual value stabilization in non-root nodes outweighs the potential performance gains.

For all configurations, the average number of component bounds per branching decision rises steeply until around depth 100, then stabilizes and gradually falls, though variation increases. This behavior can be explained as follows: in the initial levels, there are many fractional columns, but few branching decisions have been imposed, making it easy to split the columns into two groups. As branching continues, the number of fractional columns remains high, but many constraints have already been imposed, making it harder to find separating component bound sequences, requiring more component bounds. Eventually, the number of fractional columns decreases, making it easier to find separating component bounds again. The point at which this tipping occurs likely depends on the instance.

We also observe differences between using only the \texttt{MostDistinctMedian}, the \texttt{MaxRangeMidrange}, or both first-stage separation heuristics. The combination of the two peaks at the least value out of the three configurations, while the \texttt{MaxRangeMidrange} configuration peaks the highest. However, it is also the fastest to drop back to an average of just over one, while the \texttt{MostDistinctMedian} configuration hovers at its peak for over 200 depths before decreasing.

The narrow confidence interval until depth 100, despite significant changes in the mean, is surprising. The sudden drop in the average number of component bounds at very shallow depths, followed by a steep rise, remains unexplained.


\cleardoublepage
\chapter{Conclusion}\label{ch:conclusion}
In this thesis, we have formalized and explored constraints that exist solely within the reformulated problem in column generation, where their validity in the master problem is ensured only by specific modifications to the pricing problem. To address these needs, we have extended the \GCG{} solver by introducing an interface that facilitates the creation and management of these specialized constraints, termed \texttt{generic mastercuts}. This new interface is designed to streamline the implementation of advanced branching rules and separators that rely on conditions specific to the reformulated problem.

Leveraging this interface, we have implemented the component bound branching rule, as presented by Desrosiers et al. \cite{thebook}. This rule offers a simpler alternative to Vanderbeck's generic branching scheme \cite{vanderbeck2011branching} for branching on component bound sequences. Our evaluation reveals that, while Vanderbeck's scheme significantly outperforms the new component bound branching rule, this advantage is largely due to the preservation of the pricing problem's structure. This preservation allows the continued use of specialized optimization algorithms throughout the entire search tree. When Vanderbeck's scheme is forced to rely on a generic \MIP{} solver for pricing, as is required by the component bound branching rule, the latter actually performs better on average. This finding underscores the critical importance of maintaining the structure of the pricing problem to enhance the performance of the branch-and-price algorithm, thereby emphasizing the need for careful selection of the decomposition strategy.

Although the component bound branching rule does not substantially improve the overall performance of \GCG{}, its implementation highlights the versatility and potential of the new generic mastercut interface. This interface makes it feasible to implement complex branching rules or separators that operate exclusively within the reformulated problem. As a further refinement, expanding the interface to accommodate constraints that do not require additional variables in the pricing problem, such as those in Vanderbeck's generic branching, would broaden its applicability and utility.

Looking ahead, future research could focus on developing new branching rules and master separators that can fully exploit the capabilities of this interface. This work will likely open up new avenues for improving the efficiency and effectiveness of branch-price-and-cut based algorithms in a wider range of applications. Moreover, given the importance of choosing the right component bound sequence for branching, as evidenced by our evaluation, further work could explore more effective heuristics for identifying these sequences. Investigating whether a theoretically optimal component bound sequence exists or developing more sophisticated heuristics could significantly enhance the performance of the component bound branching rule and, by extension, the efficiency of branch-and-price algorithms in solving large-scale integer programs.

\cleardoublepage

% References
\printbibliography

\end{document}

% End of document
\end{document}